% !TEX encoding = UTF-8 Unicode
% !TEX TS-program = xelatex
IMAT\\
Analyse dp\\
Application mécanique, optimisation probabilités, géométrie algébrique, codage orthographique, informatique. Numérique.\\
imath.fr\\
Buchitte Guy,
google scholar

\title{Analyse convexe et dualité en optimisation}
1. Principes généraux en optimisation
2. Analyse convexe
3. Problèmes en dualité (en dimension infinie)
4. Application en transport optimal

Problèmes du calcul des variations.
\chapter{Principes généraux en optimisation}
\section{initiation}
On considère des problèmes du type: $\inf\{F(u)|u\in X\}$, où $X$---espace vectoriel (Banach de dimension infinie)
$F:u\in X\rightarrow  ]-∞,+∞]$
\begin{itemize}
	\item Existence d'un minimisation ù
	\item Unicité?
	\item Conditions d'optimalisés
\end{itemize}
Nécessaire ou suffisante ou des deux.
Si dom $F≠ø$.
\begin{remark}
	Il existe une suite $(u_n)$ dans $X$ telle que 
	$\lim_{n\to∞}F(u_n)=\inf_X F$
\end{remark}
\begin{notations}
	$\dom F=\{u\in X|F(u)<+∞\}$.
\end{notations}
Si dom $F≠ø$ $\implies$ $\inf F<+∞$. $\forall α>\inf_XF \exists u\in X$ tel que $\inf_XF≤F(u)<α$. $(u_n)$ est une suite minimisante.

\section{Cas $X=\R^N$} % (fold)
\label{sec:cas_x_r_n}

% section cas_x_r_n (end)Cas $X=\R^N$. 
Soit $f:\R^N\rightarrow \R$ continue et $C$ une partie \underline{fermée, non vide} de $\R^N$.
\begin{theorem}
	$f$ atteint son minimum sur $C$ sous l'une des conditions suivantes:
	\begin{enumerate}[(i)]
		\item $C$ est bornée
		\item $\lim_{\norm{x}\rightarrow +∞}f(x)≤+∞$ (coercitive)
	\end{enumerate}
\end{theorem}
\begin{proof}
	Soit $(x_n)$ une suite minimalé i.e. $x_n\in C$ et $f(x_n)\rightarrow  α:=\inf_Xf$.
	
	Cas i) $x_n\in C$ compact $\implies$ $\exists x_{n_k}$, $\exists \bar x | x_{n_k}\to\bar x$.
	Alors $α=\lim_{n\to∞}f(x_{n_k})=f(\bar x)$ par continuité de $f$ au pt $\bar x$
	
	Cas ii) Soit $β>α$. Puisque $f(x)\to+∞$ qd $n\norm{x}\to∞$ : $\exists R>0 | f(x)>β si \norm{x}≥R$.
	D'autre part $\exists N|\forall n≥N f(x_n)≤β$. Donc on a $\norm{x_n}≤R\ \forall n≥N$ et la suite $(x_n)$ est bornée.
	Donc $\exists x_{n_k}$, $\exists \bar x\in \R^N | x_{n_k}\to \bar x$.
	$x_{n_k}\in C$ fermé et $x_{n_k}\to \bar x$ $\implies$ $\bar x\in C$.
	donc on a: $\bar x\in C$ et $f(x)=\lim_{k\to ∞} f(x_{n_k})=α=\inf_C f$
	Bien on a: $F:X\rightarrow ]-∞,+∞]$
\end{proof}
$F(x)= f(s)$ si $x\in C$ et $+∞$ sinon, $X=\R^n$,
$\inf_C f=\inf_X F$.
\begin{remark}
	Ici $F$ n'est pas continue mais dans les cas i) et ii), on a $\lim_{\norm{x}\to∞}F(x)=+∞$. En optimisation $f$ s'appelle le critère et $C$ est la contrainte.
	$F=f+δ_C$. Où $δ_C = 0$ si $x\in C$ et $+∞$ sinon.
\end{remark}
Mais $F$ vérifie la propriété $x_n\to x$ $\implies$ $\liminf_{n\to ∞} F(x_n)≥F(x)$
($\liminf_{n\to∞}F(x)=\liminf_{n\to ∞}[f(x_n)+δ_C(x_n)]≥\underbrace{\liminf_{n\to∞} f(x_n)}_{f(x)}+\underbrace{\liminf_{n\to ∞}δ_C(x_n))}_{≥δ_C(x)}$)
$\liminf_{n\to +∞} δ_C(x_n)\overset{?}≥δ_C(x)$. \underline{1er} cas $\liminf_{n\to ∞}δ_C(x_n)<+∞$. $\forall N\exists n> N\ x_n\in C$ $\implies$ $\exists x_{n_k}|x_{n_k}\in C \forall k$ $\implies$ $x\in C$ ($C$ est fermé) $\implies$ $δ_C(x)=0$. \underline{2me} cas $\liminf_{n\to ∞}δ_C(x_n)=+∞$ trivial.

Autre preuve.
$x\in C$---trivial; $x\not\in C$ $\implies$ $\exists N|x_n\not\in C\ \forall n≥N$ $\implies$ $δ_C(x_n)=+∞\ \forall n≥N$. 

\begin{definition}
	$F: (X,d)\rightarrow ]-∞,+∞]$ espace métrique est \textsc{Semi-Continue Inférieure} au point $x\in X$ si $\forallα<F(x)\  \exists R>0\ |\ d(x,y)<R$ $\implies$ $F(y)>α$ (ou bien $\exists V$ ouvert contenant $x$ tel que $\inf_VF>α$)
\end{definition}
\begin{definition}
	$F$ est f.c.i sur $X$ (f.s.c.---\textsc{Fermer Semi-Continuons}) si $F$ est s.c.i. en tout point $x\in X$.
\end{definition}
\begin{lemme}
	$F$ est s.c.i. sur $X$ si et seulement si l'une des conditions suivantes est vérifié:
	\begin{enumerate}
		\item  $\forall R\in\R |\{F≤R\}$ est un fermé de $X$ ($\{F≤R\}=\{x\in X|F(u)≤R\}$)
		\item L'ensemble $\epi F=\{(u,x)\in X\times \R|F(u)≤α\}$ est fermé dans $X\times\R$.
		\item On a pour toute suite $(u_n)$ dans $X$ $u_n\to u$ $\implies$ $\liminf_{n\to ∞} F(u_n)≥F(u)$ 
	\end{enumerate}
\end{lemme}
$X=\R$
$F(x)=\left\{\mqty{0\text{, si }x≠0 \\-1\text{, si }x=0}\right.$ \\$\epi F=\{(x,α), F(x)≤α\}$. $F_n(x)=\left\{ \mqty{0\text{, si }x≤0\text{ ou } x≥\frac 1n\\1-n\abs{x}\text{, si }0≤\abs{x}≤\frac 1n}\right.$.

\begin{theorem}
	Soit $F:X\rightarrow ]-∞,+∞]$ où $X$ est un e.v.n. local compact; $F$ s.c.i. et coercive alors F atteint son minimum et l'ensemble ses solutions $\Argmin F=\{u\in X| F(x)=\inf F\}$ est un compact non vide. 
\end{theorem}
\begin{proof}
	Soit $α_n$ une suite de de réels telle que: $α_{n+1}≤α_n$. $α_n\to \inf_XF α_n>\inf_X F$. Posons $K_n=\{u\in X|F(u)≤α_n\}$. On a:
	\begin{itemize}
		\item $K_n≠ø$ (car $α_n>\inf F$)
		\item $K_{n+1}\subset K_n$ (car $α_{n+1}<α_n$)
		\item $K_n$ fermé (car $F$ est s.c.i.)
	\end{itemize}
	Posons $K=\cap_{n=1}^{n=+∞}K_n$. Si $\lim_{\norm{u}\to+∞}F(u)=+∞$, alors $\exists R>0$. $K_n\subset\{\norm{u}≤R\}$ (c'est compact de $X$) $\forall n$. (car $\exists R>0 | \norm{u}>r$ $\implies$ $F(u)>α_n \forall n$)
	$(K_n)$ et une suite $\searrow$ de compacts non vides: $K_n$ est fermé borné
	Donc $K=\cap K_n$ est donc un compact non vide. Or $K=\{u\in X| F(u)≤α_n\forall n\} =\{u\in X| F(u)≤\inf_X F\}=\Argmin F$.
\end{proof}
\begin{problem}
	$X$ Banach $\dim X=+∞$ $\implies$ $X$ non local compact;
	Idée: utiliser une topologie $G$ plus faible que la topologie de la norme et telle que:
	\begin{itemize}
		\item $F$ est $G$ s.c.i. et $\forall α|\{F≤α\}$ est $G$-compact.
	\end{itemize}
\end{problem}
\section{Cas où $X$ est un Hilbert} % (fold)
\label{sec:cas_ou_x_est_un_hilbert}
\begin{rappel}
	$X$ Hilbert avec produit scalaire $(u|v)$ $\norm{u}=\sqrt{(u|u)}$. Soit $C$ \emph{convexe} fermé non vide de $X$; $x\in X$ $f:y\in X\rightarrow \norm{x-y}$;
	$\inf_{y\in C} \norm{x-y}=d(x,C)$ distance de $x$ à $C$.
\end{rappel}
\begin{theorem}
	$\exists! x^\ast \in C$ tel que $\norm{x-x^*}=d(x,C)=\inf_{y\in C}\norm{x-y}$. (ici $F(y)=\norm{x-y}+δ_C(y)$).
\end{theorem}
\begin{remark}
	$F$ est s.c.i. et coercive. ( $\lim_{\norm{y}\to+∞}\norm{x-y}=+∞$ ) mais $X$ n'est pas local compact.
\end{remark}
\begin{proof}
	Soit $(y_n)$ une suite dans $C$ telle que $\norm{x-y_n}\to α=d(x,C)$. Alors on montre que $(y_n)$ est une suite de Cauchy en utilisant:
	$\norm{a-b}^2+\norm{a+b}^2=2\norm{a}^2+2\norm{b}^2$
	$a=\frac{x-y_n}2$, $b=\frac{x-y_m}2$.
	donc $\frac{\norm{y_n-y_m}^2}4+\norm{x-\frac{y_n+y_m}2}^2 = \frac{\norm{x-y_n}^2}2+\frac{\norm{x-y_m}^2}2\to α^2$ ($C$ convexe $\frac{y_n+y_m}2\in C$ et $\norm{x-\frac{y_n+y_m}2}≥α$)
\end{proof}

$x_1^*$, $x_2^*$ solutions $\implies$ $x_1^* + x_2^*/2$ solution.
$0≤\norm{x-\frac{x_1^*+ x_2^*}2}≤\frac 12 \norm{x-x_1^*}+\frac 12 \norm{x-x_1^*}< \frac 12 α+\frac 12 α$

$x^*$ est solution $\iff$ $\Re((x^*-x|x^*-y))≤0$ $\forall y\in C$.
$X$ espace de Hilbert sur $\R$, $a(u,v)$ forme bilinéaire symétrique: $(a(v, u)=a(u, v))$.
Telle que
\begin{itemize}
	\item $\abs{a(u,v)}≤C\norm{u}\norm{v}$ (continuité) $\forall(a,v)\in X\times X$
	\item $\exists k>0$ $a(u,u)≥k\norm{u}^2$ $\forall u\in X$
	\item $f$ une forme linéaire continue sur $X$ ($f\in X^*$) (notation $\expval{f,u}$ au bien de $f(v)$).
\end{itemize}
\begin{theorem}[Lax-Milgram]
	$\exists!u\in X$ tel que $a(u,v)=\expval{f,v}$ $\forall v\in X$. De plus, si on pose $F(u)=\frac 12 a(v,u)-\expval{f,v}$, on a: $F(u)≤F(v)$ $\forall v\in X$ (i.e. $F(u)=\min_X F)$ et $u$ est l'unique minimiser de $F$.
\end{theorem}
\begin{remark}
	$F$ est continue d'après i) (exo) $\lim_{\norm{u}\to ∞}F(u)=+∞$ d'après (ii) (exo) $(F(u)≥k\norm{u}^2 - \expval{f,u}≥k\norm{u}^2_X-\norm{f}_{X^*}\norm{u}_X)$ $F$ est convexe.
\end{remark}
\begin{corollary}[Stampacchia]
	Soit $C$ un convexe fermé de $X$ et $E(v)=\frac 12 a(v,v)-\expval{f,v}$ (qui est convexe, continue, convexe sur $X$)
	($F(v)=E(v)+δ_C(v)$).
	Alors $\exists! u\in C$ tel que $E(u)=\inf_{v\in C} E(v) u$ est caractérisée par l'inéquation:
	$a(u,v-u)≥\expval{f,v-u} \forall v\in C$. 
\end{corollary}
\begin{remark}
	On prenant $C=X$, on retrouve Lux-Milgran car $a(u,w)≥\expval{f,w}$ $\forall w\in X (w=v-u)$ $\implies$ $a(u,w)=\expval{f,w}$.
\end{remark}

$a:X\times X\rightarrow \R$ $f\in X^*$
$\left\{\mqty{a(u,v)≤M\norm{u}\norm{v}\\a(u,v)≥k\norm{u}^2}\right.$.

$\inf_{u\in C}\{\frac 12 a(u,u)-\expval{f,u}\}$
$C$ convexe fermé de $X$.

$\exists!\bar u\in C$ tel que $\frac 12 a(\bar u,\bar u)-\expval{f,\bar u}≤\frac 12a(v,v)-\expval{f,v}\ \forall v\in C$. Alors $a(\bar u,v-\bar u)≥\expval{f,v-\bar u}\ \forall v\in C$.

\begin{remark}
	La fonctionnelle $F(v)=\frac 12 a(v,v)-\expval{f,v}$ est continue sur $X$ (exo). Elle est convexe (même strictement convexe). Elle es coercive car $F(v)≥\frac k2\norm{v}_X^2-\norm{f}_{X^*}\norm{v}_X$ $\implies$ $\lim_{\norm{v}\to +∞}F(v)=+∞$. Si ($u_n$) est une suite minimisante  sur $C$, alors ($u_n$) est bornée dans $X$. Mais on ne peut pas extraire une sous suite ($u_{n_k}$) telle que $u_{n_k}\to u$ dans $X$ ($X$ n'est pas loc compact).
\end{remark}
\begin{proof}
	(argument analogue à celui du Thm de projection dans un Hilbert)
	
	Posons $(u|v)_a=a(u,v)$. C'est une forme bilinéaire symétrique positive: 
	$(u|u)_a≥k\norm{u}^2>0$ si $u≠0$. Donc c'est un produit scalaire sur $X$. Norme associée $\norm{u}_a=\sqrt{(u|u)_a}$.
	
	Les normes $\norm{•}$ et $\norm{•}_a$ sont équivalente car:
	$k\norm{u}^2≤\norm{u}_a^2≤M\norm{u}^2$. En particulier $(X, \norm{•}_a)$ est un espace de Hilbert. La forme linéaire.
	$L:v\in X \rightarrow  \expval{f,v}$ est continue dans $(X,\norm{•}_a)$. D'après Riesz:
	$$\exists !u_0\in X\text{ tel que } (u_0|v)_a=\expval{f,v} \forall v\in X.$$
	D'après le Thm de projection ($C$ est un convexe ferme de $(X,\norm{•})$):
	$$\exists!\bar u\in C\text{ tel que } \norm{u_0-\bar u}_a=\inf_{u\in C}\norm{u_0-v}_a.$$
	En particulier, on aura:
	$$a(\bar u-u_0,\bar u-u_0)=\inf_{u\in C}a(v-u_0,v-u_0)$$
	Or $\frac 12a(v-u_0,v-u_0) =\frac{a(v,v)}2-a(v,u_0)+\frac{a(u_0,u_0)}2=\frac 12 a(v,v)-\expval{f,v}+\frac{a(u_0,u_0)}2$ d'où 
	$\frac 12a(v,v)-\expval{f,v}≥\frac 12 a(\bar u,\bar u)-\expval{f,\bar u} \forall v\in X$. ($\frac 12 a(v-u_0,v-u_0)-\frac{a(u_0,u_0)}2≥\frac{a(\bar u-u_0,\bar u-u_0)}2-\frac{a(u_0,u_0)}2$)
	
	En fait on a:
	$\bar u\in \Argmin_CF$ $\iff$ $\bar u=\proj_Cu_0$.
	
	De plus toute suite minimisante $(u_n)$ vérifie $\norm{u_0-u_n}_d\to\inf_{u\in C}\norm{u_0-v}_a$
	et donc de Cauchy pour $\norm{•}_a$ (donc aussi $\norm{•}$)
	
	$\bar u sol$ $\iff$ $\bar u=\proj_Cu_0$ $\iff$ $(u_0-\bar u|u_0-v)_a≤0 \forall v\in C$ $\iff$ $a(u_0-\bar u|v-u_0)≤0\ \forall v\in C$ $\iff$ $a(\bar u-u_0|\bar u-v)≤0\forall v\in C$ $\iff$ $a(\bar u,\bar u-v)≤a(u_0,\bar u-v)$ $\iff$  $a(\bar u,v-\bar u)≥a(u_0,v-\bar u)$ $\iff$ $a(\bar u, v-\bar u)≥\expval{f,v-\bar u}\ \forall v\in C$.
\end{proof}
\begin{remark}
	Si la contrante $C$ est un sous espace vectoriel fermé $V$ de $X$ on obtient:
	$$\bar u\text{ minimise }\frac 12a(v,v)-\expval{f,v}\text{ sur }X \iff a(\bar u, v)=\expval{f,w}\ \forall w\in V$$
	
	Si $V=X$ on obtient l'équation $a(\bar u, w)=\expval{f,w}\forall w\in X$. ($A\bar u|w)=\expval{f,w}$ où $A$ opérateur linéaire auto adjacent-continue de $X$ dans $X$. 
	$\implies$ $A\bar u=f$
\end{remark}
\begin{rappel}
	$A\in s(X)$ ($A^*=A$.)
	$a(u,v)=(Au|v)$ bilinéaire symétrique $|a(u,v)|≤M\norm{u}\norm{v}$
\end{rappel}
\begin{example} élémentaire.
	$X=\{u\in L^2(0,1)\ |\ u'\in L^2(0,1)\}$
	$u\in X$ $\iff$ $u\in L^2(0,1)$ et $\exists v\in L^2(0,1)$ tel que:
	$∫_0^1uφ'\dd{x}=-∫_0^1vφ\dd{x}$
	
	$\forall f\in C^1(0,1)$ et $φ(0)=φ(1)=0$.
	\begin{itemize}
		\item si $u\in C^1$ on trouve $v=u'$	
		\item si $u$ est $C$ continue, $C^1$ par morceaux 
		$u = 1$ si $x<\frac12$ et $-1$ si $x>\frac 12$
		$∫_0^1uφ'=∫_0^{\frac 12}uφ'+∫_{\frac 12}^1uφ'=-∫_0^1 vφ + (u(\frac 12+0)-u(\frac 12 -0))φ(\frac 12)$.
	\end{itemize}
	$(u|v)=∫_0^1(uv+u'v')\dd{x}$
	$\norm{u}^2=∫_0^1(|u|^2+|u'|^2)\dd{x}$.
	$u\in X$ $\implies$  $\exists \tilde u=u$ pp $|\tilde u(x)-\tilde u(y)|≤\norm{u'}_{L^2(0,1)}\sqrt{|x-y|}$
	$x<y$ $|u(x)-u(y)|=|∫_x^yu'(t)\dd{t}|≤\sqrt{|y-x|}\sqrt{∫_0^1|u'|^2\dd{t}}$.
	$\inf\limits_{\substack{v\in X\\v(0)=v(1)=0}}[\frac 12∫_0^1|u'|^2\dd{x}-∫_0^1fv\dd{x}]$, $f\in L^1(0,1)$.
	
	Soit $H=\{u\in X\ |\ u(0)=u(1)=0\}$. $u_n\overset{X}{\to}$ $\implies$ $u_n\rightarrow u$ uniformément sur $[0,1]$. C'est un sous espace fermé de $X$, donc un Hilbert.
	
	Ici $a(u,v)=∫_0^1u'v'\dd{x}$\\
		\textbullet  $|a(u,v)|≤\norm{x'}_{L^2}\norm{v}_{L^2}≤\norm{u}_H\norm{v}_H$\\
		\textbullet  $a(u,u)=∫_0^1|u'|^2\dd{x}\overset{?}≥ k\norm{u}^2_H$.
		$u(x)=u(0)+∫_0^1u'(t)\dd{t}$ $\implies$ $|u(x)|≤\norm{u'}_{L^2}\sqrt{x} ≤\norm{u'}_{L^2}$ $\implies$ $∫_0^1|u(x)|^2\dd{x}≤\norm{u'}^2_{L^2}$ $\implies$ $\norm{u}_{L^2}≤\norm{u'}_{L^2}$ si $u\in H$, Donc 2 $a(u,u)≥∫_0^1 {u'}^2\dd{x}+∫_0^1u^2\dd{x}=\norm{u}_H^2$.
		
		$a(u,u)≥\frac 12 \norm{u}_H^2$
		
		Lax Milgram $\implies$ $\exists!u\in H\ |\ \frac 12$ ${u'}^2-∫_0^1fu'\dd{x}≤\frac 12∫_0^1|v'|^2-∫_0^1fv\dd{x} \forall v\in H$.
		
		Conclusion de la schuhcnu $a(u,v)=\expval{f,v} \forall v\in H$
		$∫_0^1 u'v'\dd{x}=∫_0^1 fv\dd{x}$,$ \forall v\in H (u(0)=v(1)=0)$.
		
		Supposons que la sol $u$ est 2 fois dérivable sur $]0,1[$
		$∫_0^1u'v'\dd{x}=[u'v]_0^1-∫_0^1u''v\dd{x}$ $\implies$ $ -∫u''v\dd{x}=∫_0^1 fx\dd{x} \forall v\in H$ $\iff$ $ -u''=f$
		
		Ainsi
		$$\left\{ \mqty{-u''=f\text{ sur } ]0,1[\\u(0)=u(1)=0}\right.$$
		Posons $f(x)=∫_0^1xf(t)\dd{t}$
		$-u'=F(x)+λ$.
		$u(0)=u(1)=0$ $\implies$ $∫_0^1 u'(t)\dd{t} =0$ $\implies$ $λ=-∫_0^1F(x)\dd{x}$.
		$u(x)=u(0)+∫_0^xu'(t)\dd{t}$ $\implies$ $u(x)=x∫_0^1F(t)\dd{t}-∫_0^xF(t)\dd{t}$.

\end{example}
% section cas_ou_x_est_un_hilbert (end)
\section{Cas où $X$ est un Banach (non Hilbert)} % (fold)
\label{sec:cas_ou_x_est_un_banach_non_hilbert}
On considère une topologie $C$ sur $X$ telle que
$\forall R\ \{u\in X\ \norm{u}≤R\}$ est $C$-compact.
\begin{rappel}
	$C$ est plus faible que la topologie associée a la norme.
\end{rappel}
\subsection{Cas Importantes} % (fold)
\label{sub:cas_importantes}
\begin{itemize}
	\item $X$ est un Banach réflexif ($X^{**}=X$)
	$x\in X\rightarrow \hat x\in X^{**}$ ou $\hat x(f)=f(x) \forall f\in X^*$ (évaluation de $f$ au point $x$) $\norm{\hat x}_{X^{**}}=\sup_{\norm{f}_{X^*}≤1}\hat x(f)=\sup_{\norm{f}_{X^*}≤1}f(x)=\norm{x}_X$. L'application $x\in X\rightarrow \hat x\in X^{**}$ est une isométrie.
	
	$C$=topologie faible de $X$.
	$x_n\overset{\text{faible}}{\to}x \overset{\text{def}}{\iff}\forall f\in X^*\ f(x_n)\to f(x)$
	
	\textbf{Ex}. $X$ Hilbert sur $\R$ $(•|•)$. $f\in X^*$ $\implies$ $\exists y\in X\ |\ f(x_n)=(x_n|y) $
	
	$x_n\overset{\text{faible}}{\to}x$ $\iff$ $(x_n|y)\to (x|y)\ \forall y\in X$. $x_n\overset{\text{febi}}{\to}x$ $\iff$ $x_n\overset{\text{faible}}{\to}x \oplus \limsup_{n\to ∞}\norm{x_n}≤\norm{x}.$
	
	On a toujours:
	$x_n\overset{\text{faible}}{\to}x$ $\implies$ $\liminf_{n\to∞}\norm{x_n}≥x$ La fondamentale $F(x)=\frac 12\norm{x}^2-f(x)$ est C s.c.i. pour tout $f\in X^*$.
\end{itemize}
\begin{rappel}
	Dans un Banach réflexif pour tout $R>0$, $\{x\in X\ |\ \norm{x}≤R\}$
	est faiblement-compact. De toute suite bornée $(x_n)$ on peut extraire une sous-suite $(x_{n_k})$ telle que il existe $x\in X$ tel que $x_{n_k}\overset{\text{faible}}{\to}x$ qd $k\to ∞$.
\end{rappel}
\begin{exercise}
	
	$X$ Hilbert $\{e_n,\ n\in\N\}$ borne orthonormale $\norm{e_n}=1$.

	$\forall y\in X\ (e_n|y)$ $\implies$ $\lim_{n\to∞}(y|e_n)=0 \forall y\in E e_n\rightharpoonup 0$ faiblement.
\end{exercise}
\begin{exercise}
	$X =L^2(0,1)$ $f_n(x)=\sin(2πnx)$
	
	$(f_n|g)=∫_0^1f_n(x)g(x)\dd{x}\to ∫_0^1fg$
	
	$|∫_0^1g(x)\sin(2πnx)\dd{x}|=|∫_0^1g'(x)\frac{\cos(2πnx)}{2πn}\dd{x}|≤\frac C{2πn}\to 0$.
	
	$∫_0^1g(x)\sin(2πnx)\dd{x}=\frac 12$.
	$\norm{f_n}=\frac 1{\sqrt{2}}$
\end{exercise}

$L^p(Ω)$ $Ω\subset\R^N u\in L^p(Ω)$ $\iff$ $∫_Ω |u|^p\dd{x}<+∞$.
$\norm{u}_{L^p}=(∫_Ω|u|^p\dd{x})^{\frac 1p} p réel \in [1,+∞[$.

$L^∞(Ω)=\{u:Ω\rightarrow  \R\ \exists k\ |\ (u|u)≤k\}$
$\norm{u}_{L^∞(Ω)}=\inf\{k\ |\ |u(x)|≤k pp\}$

Sii $Ω$ est borne dans $\R^N$ $L^p(Ω)\subset L^q(Ω)$ si $1≤q≤p≤+∞$. $u\in L^∞(Ω)$ $\implies$ $\norm{u}_{L^∞(Ω)}=\lim_{q\to ∞}\norm{u}_{L^q(Ω)}$

$L^p(Ω)$ est Banach séparable $\forall p\in [1,+∞]$. $L^p(Ω)$ réflexif $\iff$ $1<p<+∞$.

$(L^p(Ω))^*\sim L^{p'}(Ω)$ si $p\in [1,+∞[$ et $p'=\frac p{p-1}$ ($\frac 1p+\frac 1{p'}=1$) ; $p'=∞$ si $p=1$.

$l\in (L^p(Ω))^*$ $\implies$ $\exists g\in L^{p'}(Ω)$ | $l(f)=∫_Ωfg\dd{x}$
$\forall f\in L^p(Ω)$, $(L^p(Ω))^{**}\sim (L^{p'})\sim L^p(Ω)$ si $1<p<+∞$.

\paragraph{$L^1(Ω)$ n'est pas réflexif} % (fold)

$Ω=\R$
$u_n(x)=\left\{\mqty{n \text{ si } 0≤x≤\frac 1n\\0\text{ sinon}}\right.$
$\norm{u_n}_{L^1(\R)}=1$.
$u_n\to u$ faiblement dans $L^{1}(\R)$ si (définition) 
$\forall v\in L^∞$ $∫_0^1u_nv\dd{x}\to ∫_0^1uv\dd{x}$. Soit $v$ continue sur $[0,1]$.

$∫_0^1u_nv\dd{x}=n∫_0^{\frac 1n}v(x)\dd{x}$. Donc $∫_0^1u_nv\dd{x}\to v(0)$.
$\expval{δ_0,v}=v(0)$. Si $u$ existe, on doit avoir $u(0)=∫uv\dd{x}$.

2eme cas important. $X$ est le dual d'un espace de Bansch séparable $Y$. $X=Y^*$ (ex. $X=L^∞(Ω)$, $Y=L^1(Ω))$)(ex. $X=M_b(\R)$, $Y=C_0(\R)$)

On choisit pour $C$ la topologie *-faible
Soit $(f_n)$ suite dans $X^*$.
\begin{definition}
	$f_n\overset{*}{\to}f$ $\iff$ $\forall x\in X\ f_n(x)\to f(x)$.
\end{definition}

\begin{theorem}
	$\norm{f_n}_{X^*}≤M\ \forall n$ $\implies$ $\exists f_{n_k}$,$ \exists f\in X$ tel que $f_{n_k}\overset{*}{\to}f$.
\end{theorem}
\begin{example}
	Soit $(u_n)$ une suite dans $L^∞(Ω)$ ( $Ω$ ouvert de $\R^N$). Telle que $|u_n(x)|≤M$ pp $x\inΩ$, $\forall n\in \N$. Alors $\exists u\in L^∞(Ω)$, $\exists u_{n_k}\ |\ \lim_{k\to ∞}∫_Ωu_{n_k}v\dd{x}=∫_Ωuv\dd{x}\ \forall v\in L^1(Ω)$.
\end{example}
\begin{example}
	Soit $(ψ_n)$ une suite de mesures positives bornées sur $[0,1]$. Alors $\exists ψ$ mesure borne sur $[0,1]$ telle que
	$∫_0^1φ\dd{ψ}\to ∫_0^1φ\dd{ψ}$ $\forall φ$ continue sur $[0,1]$
	$ψ_n=f_n\dd{x} ψ=δ_0$
	
	$ψ_n\overset{*}{\to}δ_0, ψ$
\end{example}

$X$ Banach $G$ \texttt{faible} si $X$ réflexif et \texttt{*faible} si $X=Y^\perp Y$ Banach séparable.
Propreté $\norm{u_n}_X≤C$ $\implies$ $\exists u\in X \exists u_{nk}\ |\ u_{nk}\overset{G}{\to} u$ 

Notations: $\left\{\mqty{u_n\rightharpoonup u \text{ faible}\\ u_n\overset*\rightharpoonup u \text{ *faible} }\right.$

\begin{theorem}
	Soit $F:X\rightarrow  ]-∞,+∞]$ telle que: 
	\begin{enumerate}
		\item $F$ est $G$ s.c.i. (et $\exists u_0\in X\ F(u_0)<+∞$)
		\item $\lim_{\norm{u}\to ∞}F(u)=+∞$
	\end{enumerate}
	Alors $\Argmin F$ est un $G$-compact non vide de X.
\end{theorem}
\begin{proof}
	(identique au cas $X$ local compact)
\end{proof}
\begin{remark}
	La propriété i) est souvent difficile à établir même si $F$ est continue par la topologie forte de $X$.
\end{remark}
\begin{example}
	$X=\{u\in C^0([0,1])\ |\ u(0)=u(1)=0\ |\ u'\in L^2(0,1)\}$
	$I=]0,1[$ $\norm{u}_X=\sqrt{∫_0^1|u|^2+|u'|^2\dd{x}}$ ( ou bien $\norm{u}=(∫_0^1 |u'|^2\dd{x})^{\frac 12}$) ($|u(x)|≤∫_0^1 |u'(x)|\dd{x})≤\norm{u'}_{L^2}$ $\implies$ $\norm{u}_{L^2}≤\norm{u'}_{L^2}$)
	
	$X$ est Hilbert noté $H_0^1(I)$ (ou bien $W_0^{1,2}(I)$)
	
	Choix de $G$. $G$ topologie faible ($X$ est réflexif)
	(ou $G$ topologie associe a la convergence uniforme $u_n\overset{G}{\to} si \sup_I|u_n-u|\to 0$)
	
	$u_n\rightharpoonup u$ faiblement $\iff$def $\left\{\mqty{u_n\rightharpoonup u\ L^2\mbox{-faible}\\u'_n\rightharpoonup u'\ L^2\mbox{-faible}}\right.$
	
	$\overset{Rellich}\implies$ $u_n\to u$ uniformément.
	$|u_n(x)-u_n(y)|≤\sqrt{|y-x|}\norm{u'_n}_{L^2} $ 
\end{example}

\begin{theorem}[Ascoli]
	$u_n$ continue sur un compact équicontinue et $\forall x (u_n(x))$ bornée dans $\R$ $\implies$ $\exists u_{nk}$, $\exists u$ continue | $u_{nk}\to u$ uniformément.
\end{theorem}

$\inf_{u\in X}F(u) où F(u)=∫_0^1 |1-{u'}^2|\dd{x}+∫_0^1 u^2\dd{x}$\\
	\textbullet On a bien que $\lim_{\norm{u}_X\to +∞}F(u)=+∞$
	\begin{align*}
		∫_0^1 |(u')^2-1|\dd{x}&≥∫_0^1|u'|^2-1\\
		&≥\norm{u}_X^2 -1
	\end{align*}
	\textbullet $F$ est elle faiblement s.c.i. Soit $φ$ une fonction 1-périodique de classe $C^1$ telle que $φ(0)=φ(1)=0$ (par exemple $φ(x)=\sin 2πx$).
	Soit $u_n=\frac 1nφ(nx)$. Alors $u_n(0)=u_n(1)=0 (φ(0)=φ(1)=0)$.
	$u'_n=φ'(nx)$
	$∫_0^1|u_n|^2\dd{x} =\frac{1}{n^2}∫_0^1|φ(nx)|^2\dd{x}≤\frac{C}{n^2}$ où $C=\sup|φ^2|$ $\implies$ $u_n\to 0$ uniformément et dans $L^2(I)$.
	
	$∫_0^1|u'_n|^2\dd{x}=∫_0^1|φ'(nx)|^2\dd{x}=∫_0^1|φ'(y)|^2\dd{y}$
	
	Donc $(u_n)$ est bornée dans $X$. Soit $(u_{nk})$ une sous-suite telle que $u_{nk}\overset{G}{\to}u$ ($G$=faible sans $X$)
	
	Alors on a $u=0$ d'où $u_n\overset G\to 0$ ($u_n'\to 0$ dans $L^2(I)$ faible)


\begin{example}
	$ψ:\R\to\R$ 1 périodique $ ∫_0^1|ψ|^2\dd{x}<+∞$. Alors $ψ_n(x)=ψ(nx)$ est une suite bornée dans $L^2(0,1)$ et bornée dans $L^1(0,1)$ et $ψ_n\to c$ faiblement dans $L^2(0,1)$ où $c=∫_0^1 ψ(y)\dd{y}$. En particulier si $ψ=φ'$ où $φ$ est 1-périodique, on a $ψ_n\to 0$ car $∫_0^1 ψ(y)\dd{y}=∫_0^1 φ'(y)\dd{y} = φ(1)-φ(0)=0$.
	Conclusion $u_n\rightharpoonup 0$ faiblement dans $X$. Calculons:
	\begin{align*}
		\liminf_{n\to ∞}F(u_{n})&=\liminf_{n\to ∞}∫_0^1|1-(u'_n)^2|^2\dd{x}\\
		&=\liminf_{n\to ∞}∫_0^1 |1-φ^2|\dd{y}
	\end{align*}
	$F(0)=∫_0^1|1-0|^2\dd{x}=1$


	Si $F$ était $G$ s.c.i., on aurait:
	$∫_0^1|1-(φ')^2|\dd{x}≥1$.
	$\forall φ$ 1-périodique avec $φ(0)=φ(1)=0$.

	Impossible (prendre $φ=\frac 12 -|\frac 12 -x|$).

	\textbullet $\inf_{u\in X}F(u)=0$

	$u_n=\frac 1nφ(nx)$ où $φ(x)=x$ si $x<\frac 12$ et $1-x$ si $x>\frac 12$
	sur la période $[0,1]$

	Alors $F(u_n)\to 0$ qd $n\to ∞$ car $∫_0^1|1-|u_n'|^2|\dd{x}=0$ et $∫u^2_n=\to 0$.

	Puisque $F≥0$, on a donc $\inf_XF=0$. L'infinum n'est pas atteint car
	$F(u)=0$ $\implies$ $∫_0^1|1-{u'}^2|\dd{x}+∫_0^1|u|^2\dd{x}=0$ $\implies$ $u=0$ pp et $u'=±1$ pp impossible. Donc $F(u)>0\ \forall u\in X$.

	Raidon Théorique
\end{example}
\begin{example}
	$X=H_0^1(0,1)$
	$F(u)=\frac 12∫_0^1|u'|^2+∫_0^1g(u)\dd{x}$
	où $g:\R\rightarrow [0,+∞]$ est s.c.i.
	
	$G =$ topologie faible de $X$
	($u_n\overset{G}\rightharpoonup u$ $\implies$ $\left\{ \mqty{u_n'\to u'\text{ faible $L^(0,1)$}\\ u_n\to u\mbox{ uniformément}} \right.$)
	
	où $g:\R\to [0,+∞]$ est s.c.i.
	
	Alors $F$ vérifie:
	\begin{enumerate}[i]
		\item $G$ s.c.i.
		\item $\lim_{\norm{u}\to+∞}F(u)=+∞$ ($F(u)≥\frac 12\norm{u'}^2_{L^2}≥\frac 14\norm{u}_X^2$)
		
		Soit $u_n\overset{G}\to u$. Alors $u_n'\to u'$ faible $\implies$ $\liminf_{n\to ∞}∫_0^1|u'_n|^2≥∫_0^1|u'|$. ($\liminf\norm{u'_n}≥\norm{u'}$)
		
		$\implies$ $\liminf_{n\to ∞}∫_0^1g(u_n)\dd{x}≥∫_0^1(\liminf_n(g(u_n)))\dd{x}≥∫g(x)\dd{x}$
		(car $\liminf_{n\to ∞}g(u_n)≥g(u)$).
	\end{enumerate}
\end{example}
\begin{example}
	$Ω$ ouvert convexe et fermé de $\R^d$
	
	$k(x)$ continue: $\bar Ω\rightarrow ]0,+∞[$. Soit $a,b\inΩ$.
	$\sup\{u(b)-u(a)\ |\ u\in Lip(Ω) |\nabla u(x)|\overset{pp}≤k(x)\text{ sur }Ω\}:= M_{k,Ω}(a,b)$
	
	Soit $γ(t):[0,1]\rightarrow Ω$ | $γ(0)=a et γ(1)=b$
		\begin{align*}
			u(b)-u(a)=u(γ(1))-u(γ(0)) = u(γ(t))|_0^1 &= ∫_0^1(uºγ)'(t)\dd{t}\\
			&=∫_0^1\nabla u(γ(t))•γ'(t)\dd{t}\\
			&≤∫_0^1k(γ(t))|γ'(t)|\dd{t}\\
			&≤C∫_0^1|γ'(t)|\dd{t}\\
			&≤CL(γ)	
		\end{align*}
	où $C=\sup_{\bar Ω}k(x)$
	
	$L(γ)=$longer de la courbe
	
	Donc $M_{k,Ω}(a,b)<+∞$ s'il existe une courbe de longueur finie joignant $a$ à $b$.
	
	On considère $X=Lip(Ω)=\{u\text{ continue, }u(b)=0\ |\ \nabla u\in L^∞(Ω)\}$
	
	$F:u\in X\rightarrow \left\{\mqty{u(a)\text{ si }|\nabla u|≤k|x|\text{ et }u(b)=0\\ +∞\text{ sinon}}\right.$
	
	$\inf_XF(u)=\inf\{u(a)\ |\ |\nabla u|≤k\text{ pp sur $Ω$ }u(b)=0\}=-M_{k,Ω}(a,b)$
	
	$\norm{u}_X\overset{\text{def}}= \norm{\nabla u}_{L^∞(Ω)}$
	
	(si $\nabla u=0$ pp $\implies$ $u=\const$ $\implies$ $u=0 $(car $u(b)=0$))
	
	Choix de $G$
	
	$\norm{u_n}_X≤C$ $\implies$ $\norm{\nabla u_n}_{L^∞}≤ C L^∞(Ω)=(L^1(Ω))^*$
	
	Alors $\forall x |u_n(x)-u_n(y)|≤C|x-y|$ pour $y$ racine de $x$
	
	$\implies$ $(u_n)$ est équicontinue au point $x$
	
	$u_n(b)=0$ $\implies$ $|u_n(x)|≤CL(γ)$ où $φ$ est une courbe joignant $b$ au point $x$.
	
	Alors d'après Ascoli: $\exists u_{uk}$ $\exists u$ continue | $u_{uk}\to u$ uniformément sur $\bar Ω$
	$\norm{\nabla u_{nk}}_{L^∞}≤M$ $\implies$ $\nabla u_{nk}\overset*\rightharpoonup u$ dans $L^∞(Ω)$ faible.
	
	$G$= converge uniforme sur $\bar Ω$.
	\begin{enumerate}
		\item $F(u)<+∞$ $\implies$ $\norm{u}_X≤\sup_Ωk=M$ $\implies$ $\lim_{\norm{u}\to∞}F(u)=+∞$
		\item $F$ est $G$ s.c.i. 
	\end{enumerate}
	Soit $(u_n)$ une suite telle que $u_n\to u$ uniformément et telle que $\liminf_{n\to∞} F(u_n)<+∞$ Il faut montrer que $\liminf F(u_n)≥F(u)$. Quitte à extraire une sous suite, on peut supposer que $\liminf_nF(u_n)=\lim_{n\to∞}F(u_n)$. Alors $F(u_n)$ est majoré pour $n$ assez grand et donc $\sup_n\norm{u_n}_X<+∞$. Donc on a $u_n\to u$ uniformément $|\nabla u_n|≤M$.
	
	On a donc $u_n(a)\to u(a)$.
	Montrons que $u\in X$ et $|\nabla u(x)|≤k(x)$ pp sur $Ω$.
	
	On sait que $\nabla u_n\overset{*}\rightharpoonup\nabla u$ dans $L^∞(Ω)$
	
	$\implies$ $∫_Ω\nabla u_n•v(x)\to ∫_Ω\nabla u(x)v(x)$
	
	$\forall v\in (L^1(Ω))^d $
	$v(x)=z\ind_{B(x_0,ε)} z\in\R^2$
	
	\textbullet $B(x_0,ε)$ boule centre dans $Ω$
	$\implies$ $∫_{B(x_0,ε)}(\nabla u_n(x),z)\to∫_{B(x_0,ε)}\nabla u(x),z$
	
	$|∫_{B(x_0,ε)}\nabla u_n(x) z|≤∫_{B(x_0,ε)}\nabla |u_n(x)| |z|\dd{x}≤(∫_{B(x_0,ε)}k(x)\dd{x})|z|$
	
	d'où qd $n\to ∞$
	$|∫_{B(x_0,ε)}\nabla u(x) z|≤(∫_{B(x,ε)}k(x))|z|$
	
	$\implies$
	$\frac{1}{B(x_0,ε)}|∫_{B(x_0,ε)}\nabla u_n(x) z\dd{x}|≤$
	$\frac{1}{B(x_0,ε)}|∫_{B(x_0,ε)}k(x)\dd{x} |z|$
	
	D'après le Thm des points de Lebesgue pp $x_0\inΩ \lim_{ε\to 0}\frac1{B(x_0,ε)}∫_{B(x_0,ε)}\nabla u(x) z=\nabla u(x_0) z$
	
	$\lim_{ε\to 0}\frac 1{B}∫_{B(x_0,ε)}k(x)\dd{x}=k(x_0)$
	
	$\nabla u(x_0) z≤k(x_0) z$
	$\forall z\in\R^d pp x_0\in Ω$ $\implies$ $|\nabla u(x_0)|≤k(x_0) pp x_0\in Ω$
	
	Conclusion $u_n(a)\to u(a)$ $u_n(b)$ $\forall x$ $\implies$ $u(b)=0$
	$|\nabla u|≤k(x) pp x\inΩ$
	
	Donc $u\in X$, avec $|\nabla u|≤k$ pp et $F(u)=u(a)$.
	
	\begin{rappel}
		$F(u_n)=u_n(a)$. Donc $F(u_n)\to F(u)$
		$F(u_n)\to α$, $α<+∞$ $u_n\rightharpoonup u$ uniformément $α=F(u)$
	\end{rappel}
	Ainsi $F$ est $G$ s.c.i. d'où existence d'une solution $u\in X$.
	On a:
	$M_{k,Ω}(a,b)=\sup\{u(b)-u(a)\ |\ |\nabla u|≤k sur Ω\}$
	$=\inf(∫_0^1 k(γ(t))|γ'(t)|\dd{t}, γ\in Lip(0,1,Ω)\ γ(0)=a\ γ(1)=b)$
	
	On a déjà vu que
	$u(b)-u(a)≤∫_0^1 k(γ)|γ(a)|$ si $u\in X$ et $γ\in Lip([0,1],Ω)$ $γ(0)=a$ et $γ(1)=b$
	$\implies$ $M_{Ω,k}(a,b)≤\inf\{∫_0^1k(γ)|γ'|\dd{t},\ γ(0)=a\ γ(1)=b\ \}$---distance géodésique.
	
	Posons $\bar U(x)=-\inf\{∫_0^1k(γ(t))|γ'(t)|\dd{t}\ |\ γ(0)=x\ γ(1)=b\}$
	
	Alors $\bar u(b)=0 $
	$-\bar u(a)=d_{Ω, k}(a,b)$---distance géodésique entre $a$ et $b$.
	
	Si $|\nabla\bar u|≤k(x)$ pp sur $Ω$, alors
	
	$M_{Ω,k}(a,b)≥-\bar u(a)=d_{Ω,k}(a,b)$
	
	$z\nabla \bar u(x)=\lim\frac{\bar u(x+tx)-\bar u(x)}{t}$
	
	$|\bar u(x+tz)-\bar u(x)|≤∫_0^tk(x+tz)t|z|\dd{z}$
	$\implies$ $\nabla \bar u(x)z≤k(x)|z|$
	
	$k(x)=k_1, k(x)=k_2$
	
	$k_1|c-a|+k_2|b-c|$
	
\end{example}
% paragraph l_1_\xCF\x89_n_est_pas_reflexif (end)  
% subsection cas_importantes (end)
% section cas_ou_x_est_un_banach_non_hilbert (end)
\chapter{Analyse Convexe} % (fold)
\label{cha:analyse_convexe}
$E$ espace vectoriel sur $\R$

\section{Ensembles et fonctions convexes} % (fold)
\label{sec:ensembles_et_fonctions_convexes}
$x,y\in E$ Notation
$[x,y]=\{x+tx+ty| t\in [0,1]\}$
$]x,y[=\{x+tx+ty| t\in ]0,1[\}$
$[x,y[=\{x+tx+ty| t\in [0,1[\}$

\begin{definition}
	ACE convexe si:
	$\forall (x,y)\in A\times A [x,y]\subset A$
\end{definition}
\begin{remark}
	A convexe $\implies$ $\frac{x+y}{2} \in A \forall (x,y)\in A$
	$\Leftarrow$ vrai si $A$ est fermé pomme topologie $G$ ($\R^d$) telle que $t\rightarrow (1-t)+ty (x,y\in E)$ conime de $\R$ dans $E$. 
\end{remark}
\begin{remark}[indication]
	$D=\{t\in [0,1]|(1-t)x+y\in A\}$
	$D$ est fermé $D\supset \{ 0,\frac12,1\} $ $\implies$ $ D\supset \{\frac k{2^n}, 0≤k≤2^n\}$
\end{remark}
Si $E$ est un e.v.n. ($G$ associée à la norme de $E$) alors
\begin{lemme}
	$A$ convexe de $E$ tel que $Å≠ø$. Alors $x\in Å$, $y\in Å $$\implies$ $[x,y[\supset Å$. De plus:
	$\bar Å=A$ et $Å$ est convexe.
\end{lemme}
\begin{theorem}[properties]
	\begin{itemize}
		\item $(A_i)_{i\in I}$ famille de convexes $\implies$ $\cup_{i\in I}A_i$ est un convexe.
		\item $A$ convexe $\implies$ $\left\{\mqty{Å\text{ convexe}\\\bar A convexe}\right.$
	\end{itemize}
\end{theorem}
\begin{definition}
	$co(A)$ enveloppe convexe de $A = \cup_{\substack{B convex\\B\supset A}}$ plus petit convexe contenant A.
	$\overline{co}(A)=\{\cup B|B convexe fermé B\supset A\}=\overline{co(A)}$.
\end{definition}

Si $\dim E<∞$ ($E=\R^N$) Carateodory.

$A$ fermé $\subset E$. $\overset{co(A)}=\{∑_{i=1}^{N+1}λ_ix_i | λ_i≥0,\ ∑_{i=1}^{N+1} λ_i=0\ x_i\in A\}$
compinaison convexe des vecteurs $x_1,...,x_{N+1}$.

$N=2,$
$A={a,b,c}$
$\forall x\in co(A) \exists !λ_1λ_2λ_3 | λ_1a+λ_2b+λ_3c=x, λ_1+λ_2+λ_3=1$
coordonnées barycentriques

% section ensembles_et_fonctions_convexes (end)
\section{Fonctions convexes} % (fold)
\label{sec:fonctions_convexes}
$f:E\rightarrow ]-∞,+∞]$

\begin{definition}
	$f$ convexe si $\forall(x,y)\in E\times e \forall λ\in [0,1]$:
	$f((1-λ)x+λy)≤(1-λ)f(x)+λf(y)$ (inegalite dans $]-∞,+∞]$) $f$ est strictement convexe si de plus si l'égalité entraine que $x=y$ ou bien $λ\in\{0,1\}$. 
\end{definition}

Alors
$x+y$ $\implies$ $f(\frac{x+y}2)<\frac 12[f(x)+f(y)]$

\begin{example}
	$E $e.v.n. $f_p(x)=\frac 1p\norm{x}^p$ $1≤p<+∞$
	Alors $f_p$ est convexe $\forall p\in [1,+∞[ $strictement convexe si $p>1$\\
	\textbullet  $p=2$ $\norm{\frac{x+y}2}^2=\frac 12(\norm{x}^2+\norm{y}^2)-\frac 12\norm{x-y}^2$\\
	\textbullet  $p=1$ $y=2x$ $\norm{\frac{x+y}2}=\frac 32\norm{x} \frac 12(\norm{x}+\norm{y})=\frac 32\norm{x}$\\
	\textbullet  $p=+∞$ $f_{∞}(x)=\{\substack{0 si \norm{x}≤1\\+∞ si \norm{x}>1}$ convexe sci $E\rightarrow [0,+∞]$
\end{example}
\begin{lemme}
	$f$ convexe $\iff$ $epi f=\{(x,α)\in E\times \R\ f(x)≤α \}$ est un convexe de $E\times\R$;
	$f $convexe sci $\iff$ $epi f$ est un convexe fermé de $e\times \R$.
\end{lemme}
\begin{remark}
	$f$ convexe $\implies$ $\forall α\in\R \{x\in E\ |\ f(x)≤α\}$ est un convexe de $e$.
	$\Leftarrow$ faux, mais vrai si $f$ ne prend que les valeurs $0$ et $+∞$)
\end{remark}
\begin{example}
	$δ_A: x\in E\rightarrow \bra{\substack{0 si x\in A\\+∞ si x\not\in A}}$ (indicatrice de l'ensemble $A$.)
	$δ_A$ convexe $\implies$ $A$ convexe .($epi(δ_A)=A\times \R^+$)
\end{example}
% section fonctions_convexes (end)
\section{Continuité des fonctions convexes} % (fold)
\label{sec:continuite_des_fonctions_convexes}
$E$ e.v.n. sur $\R$.
$f: E\rightarrow ]-∞,+∞]$ convexe
\begin{definition}
	$\dom f=\{x\in E\ |\ f(x)<+∞\}$.
\end{definition}
\begin{lemme}
	Soit $V$ un ouvert de $E$ tel que $\sup_V f<+∞$. Alors $f$ est continue, localement Lipstitzienne) sur $V$.
\end{lemme}
\begin{proof}
	Soit $x_0\in V$ et $M=\sup_V f$ ($M<+∞$). Quitte à ecrire fo sous la forme $f(x)=f(x_0)+g(x-x_0)$ ; on peut supposer que $x_0 = 0_E$ et $f(x_0)=0$.
	\textbullet  $V$ ouvert contenant $0$ $\implies$ $\exists R>0 | \norm{x}≤R$. $f(x)≤M$
	
	Alors pour tout $x\in E$ telle que $\norm{x}<r$ (avec $r<R$) on a:
	$f(\frac Rrx)≤M$ $\implies$ $f(\frac rR(\frac{Rx}r)≤convexite= \frac rRf(\frac{Rx}{r})+(1-\frac rR)f(0)$$\implies$ $f(x)≤\frac rR M$, si$\norm{x}<r$.
	$f(x)-f(0)≤\frac{\norm{x}R}M. \forall\norm{x}<R$
	$\implies$$\limsup_{\norm{x}\to 0}f(x)≤0=f(0_E)$
	
	Montrons que
	$\liminf_{\norm{x}\to 0} f(x)≥0$
	
	$z=-kx$
	$(1-λ)x+λz=0$ $\implies$ $x=λ(x-z)=λ(1+k)x.$ $λ=\frac1{1+k}$, $λ\in]0,1[$
	
	$\frac k{1+k} x+ \frac z{1+k}=0$ $\implies$ $f(0_e)=0≤\frac{k}{1+k}+\frac 1{1+k}f(z)$ $\implies$
	$0≤\frac k{1+k}f(x)+\frac m{1+k}$ $\implies$ $f(x)≥-\frac Mk$ si $\norm{-kx}≤R$ $\implies$ $f(x)≥-\frac{M}{\frac{R}{\norm{x}}}=-\frac MR\norm{x}$
	
	Finalement on a obtenu:
	$|f(x)-f(0_E)|≤\frac MR\norm{x}$, $\forall x\in B_R$.
	
	d'où $\lim_{\norm{x}\to 0}=f(0_E)$
\end{proof}
\begin{corollary}
	\begin{enumerate}
		\item $f$ convexe majorée sur toute boule de $E$ $\implies$ $f$ continue sur $E$.
		\item $f$ est continue sur $\overset{º}{\dom f}$ si $f $est s.c.i. 
	\end{enumerate}
	$\dom f=\cup_{n\in\N}\underbrace{\{x\in E\ |\ f(x)≤n\}}_{\text{fermée}}$.
	
	Baire $\overset{º}{\dom f≠0}$ $\implies$ $\exists n_0\ |\ \overset{º}{\{x\in E\ |\ f(x)≤n\}}≠ø$, $\forall n≥n_0$.
	Alors $x\in\overset{º}{\dom f}$ $\implies$ $\exists n\ |\ x\in V_n=\overset{º}{\{f≤n\}} f$ majorée sur l'ouvert $V_n$ $\implies$ $f$ continue au pt $x$. 
\end{corollary}
% section continuite_des_fonctions_convexes (end)

\section{Conjuguée de Fenchel} % (fold)
\label{sec:conjuguee_de_fenchel}
X e.v.n. sur $\R$
$f:X\rightarrow ]-∞,+∞]$ $\dom f=\{x|f(x)<+5\}≠ø$.
\begin{definition}
	$f*:X^*\rightarrow \bar\R$
	$\forall x^*\in X^*\qand f^*(x^*)=\sup\{\expval{x,x^*}-f(x)\ |\ x\in X\}$.
\end{definition}
\begin{remark}
	Si $E$ est un Hilbert ($X^*\sim X$), $f^*$ s'appelle la conjuguée de Fenchel de $f$. On a : $\forall x\in X\forall x^*\in X^*\qand \expval{x,x^*}≤f(x)+f^*(x^*)$ inégalité de Fenchel.
	
	\textbullet  Interprétation si $X=\R$
	$y=f'(\bar x)(x-\bar x)+f(\bar x)$ 
	ordonné de l'intersection avec $x=0$ de la tangente au graphe de f dont la ponte est égale à $x^*$ (il faut une tangente en dessous du graphe)
\end{remark}
\begin{example}
	\begin{itemize}
	\item Economie $X=\R^n$
	$f(x_1,x_2,...,x_n)$ cout de fonction de quantités $x_1, x_2,...,x_n$ de produits $A_1, A_2, ..., A_n$. $x^*=(x_1^*,x_2^*,...,x_n^*)$, $x_i^*$ $= $prix de vente unitaire de $A_i$.
	
	Profit $=x_1x_1^*+...+x_nx_n^*-f(x_1,x_2,...,x_n)$ $f^*(x^*)$=profit optimal.
	$X$ Hilbert $f_p(x)=\frac 1p\norm{x}^p$ $1≤p<+∞$.
		$f_1^*(x^*)=\sup_{x\in X}(x|x^*)-\norm{x}$ 
		$f^*(x^*)≥t\norm{x^*}^2-t\norm{x^*}≥t\norm{x^*}(\norm{x^*}-1) \forall t≥0$.
		
		$f^*(x^*)=+∞ si \norm{x^*}≤1$
		
		Si $\norm{x^*}≤1$, on a: $(x|x^*)≤\norm{x}$ $\implies$ $(x|x^*)-\norm{x}≤0, \forall x\in X$. $\implies$ $f^*(x^*)=0.$
		
		$f_1^*(x^*)=\sup_{x\in X}((x|x^*)-\norm{x})=\left\{\mqty{0 si \norm{x^*}≤1\\+∞ si \norm{x^*}>1}=f_∞(x^*)\right.$ où $f_∞=\left\{\mqty{0 si \norm{y}≤1\\+∞ si \norm{y}>1}=f_∞(x^*)\right.$
		
		$1<p<+∞$
		$f_p^+(x^*)=\sup_{x\in X}\{(x|x^*)-\frac 1p\norm{x}^p\}$
		$\forall x\in X, \exists!t≥0, \exists!u\in X \norm{u}=1 $tel que $x=tu$
		
		($t=\norm{x}$ et $u=\frac 1{\norm{x}}$)
		
		$f_p^*(x^*)=\sup_{\substack{t≥0\\\norm{u}=1}}|t(u|x^*)-\frac 1p|t|^p|=\sup_{\norm{u}=1}\sup_{t≥0}|t(u|x^*)-\frac 1p|t|^p|$
		
		$φ(t)=t(u|x^*)-\frac 1pt^p$
		$φ'(t)=(u|x^*)-|t|^{p-2}t$
		
		$φ'(\bar t)=0$ $\iff$ $|\bar t|^{p-2}t=(u|x^*)$
		
		$\sup_{t\in\R}φ(t)=φ(\bar t)=\bar t(u|x^*)-\frac 1p|\bar t|^p=\bar t\bar t|\bar t|^{p-2}-\frac 1p|\bar t|^p=|\bar t|^p(1-\frac 1p)$.
		
		$|\bar t|^{p-1}=|(u|x^*)|$ $\implies$ $|\bar t|^p=|(u|x^*)|^{\frac p{p-1}}=|(u|x^*)|^{p'} où (\frac 1p+\frac 1{p'}=1)$ $\implies$ $ \sup_{t\in\R}φ(t)=\frac 1{p'}|(u|x^*)|^{p'}$ où p' est expression conjugué de $p$.
		
		$f_p^*(x^*)=\sup_{\norm{u}=1}\frac 1{p'}|(u|x^*)|^{p'}$
		$\sup_{\norm{u}=1}(u|x^*)=\norm{x^*}$ $\implies$ $ f_p^*(x^*)=\frac 1{p'}\norm{x^*}^{p'} f_p^*=f_{p'}$
		
		Pour $p=2$, on obtenir $f_2^*=f_2$, $f_p^*=f_{p'}$ $\forall p\in]1,+∞[$
		$f_1^*=f_∞. f_∞^*$?
		
		$f_∞^*(x^*)=\sup[(x|x^*)-f_∞(x)]=\sup_{\norm{x}≤1}(x|x^*)=\norm{x^*}=f_1$.
		
		Inégalité de Fenchel $\expval{x,x^*}≤\frac 1p\norm{x}^2+\frac 1{p'}\norm{x^*}^{p'}$ égalité $\iff$ $x=x^{p-2}x$.
	\item$ C$ convexe de $E$
	$f=δ_C(x)=\left\{\mqty{0 si x\in C\\+∞ sinon}\right.$
	$f^*(x^*)=δ-C^*(x^*)=\sup_{x\in C}(x|x^*)$
	$δ_C^*$ est convexe, sous additive $(δ_C^*(x^*+y^*)≤δ_C^*(x^*)+δ_C^*(y^*))$
	et positivement 1-homogène:
	$δ_C^*(λx^*)=λδ_C^*(x^*)\ \forall x≥0$.
	
	$δ_C^*$ est appelée fonction d'appeau de $C$. 
	\end{itemize}
\end{example}
\begin{exercise}
	$δ_c^*$ est une semi-norme si et seulement si
	$C$ fermé et $-C=C$. Ce sera une norme si de plus $0\in \dot C$.
\end{exercise}
% section conjuguee_de_fenchel (end)
\section{Biconjuguée de Fenchel} % (fold)
\label{sec:biconjuguee_de_fenchel}
$f: x\rightarrow ]-∞,+∞[$
$f^{**}:X\rightarrow \bar\R$ est définie par:
$f^{**}(x)=\sup\{\expval{x,x^*}-f^*(x^*)|x^*\in X^*\}$
\begin{example}
	$X $Hilbert $f_{p}=\frac 1p\norm{x}^p$ $1≤p<+∞$
	$\forall p\in [1,+∞]$ $f_{p}^{**}=f_p$, $f_∞^{**}=f_∞$.
\end{example}
\begin{theorem}[properties]
	[de $f^*$ et $f^{**}$]
	\begin{enumerate}
		\item $f^*$ est convexe et s.c.i. pour la topologie * faible (=faible ssi $E$ est réflexif).
		\item $\inf_Xf=-f^*(0)$
		$f≤g$ $\implies$ $f^*≥g^*$
		\item $(f_i)_{i\in I}$ famille de fonctions sur $X$
		($\inf_{i\in I}f_i)^*=\sup_{i\in I}(f_i)^*$
		\item $f^{**}≤f$ (mais ${f^{**}}^*=f^*$)
	\end{enumerate}
\end{theorem}
\begin{proof}
	1) $f^*(x^*)=\sup\{\expval{x,x^{*}}-f(x)\ |\ x\in X\} où g_x: x^*\in E^* \rightarrow  \expval{x,x^*}-f(x)$ affine * faiblement continue. $G=$topologie faible $g_x$ affine, $G$-continue. Or $g^*=\sup_{x\in X}g_x$ $g_x$ convexe, $G$---s.c.i. enveloppe superior de fonctions convexes s.c.i.
	
	$f^*(0)=\sup\{-f(x)\ |\ x\in X\}=-\inf_X f$
	
	$f≤g$ $\implies$ $\expval{x,x^*}-f(x)≥\expval{x,x^*}-g(x),\ \forall x\in X$.
	$\implies$ $f^*(x^*)≥f^*(x^*)$
	4) $(\inf_{i\in I}f_i)^*|x^*)=\sup_{x}|\expval{x,x^*}-\inf_{i\in I}f_{i}(x)|=\sup_x|\expval{x,x^*}+\sup_{i\in I}-f_i(x)|=\sup_{i\in I}\underbrace{\sup_x|\expval{x,x^*}-f_i(x)|}_{f_i^*(x^*)}$
	5) $\expval{x,x^*}≤f(x)+f^*(x^*) \forall x\in X, \forall x^*\in X^*$
	
	Donc $\expval{x,x^*}-f^*(x^*)≤f(x)$ $\implies$ $\underbrace{\sup_{x^*}\expval{x,x^*}-f^*(x^*)≤f(x)}_{f^{**}}\ \forall x\in X. g_x , G$ 
\end{proof}
\begin{lemme}
	$f^{**}(x)=\left\{\mqty{\sup\{g(x)\ |\ g affine continue g≤f\}\\-∞ s'il n'existe pas de tel mimisants g}\right.$
\end{lemme}
\begin{proof}
	Soit $g$ une fonction affine continue telle que $f≥g$. Alors $\exists x^*\in X^*$, $\exists β\in\R$ tel que $g(x)=\expval{x,x^*}-β$ et $g≤g$$\iff$ $\expval{x,x^*}-β≤f(x)$, $\forall x\in X$ $\iff$ $\sup_{x\in X}(\expval{x,x^*}-f(x))≤β$ $\iff$ $f^*(x^*)≤β$ ($\implies$ $x^*\in \dom f^*$). Si $\dom f^*≠ø$ ($f^*=+∞$) $\implies$ $\not\exists g$ affine continue $≤f$.
	$\dom f^*≠ø$ $\iff$ $\exists g$ affine continue $≤f$
	Supposons $\dom f^*≠ø$ alors $\sup_{\substack{g≤f\\g affine}}g(x)=\sup_{(x^*,β)\in E^*\times\R}\{\expval{x,x^*}-β\ |\ f^*(x^*)\}$
	$=\sup_{x^*\in\dom f^*}|sup_{β≥f^*(x^*)}\expval{x,x^*}-β|=\sup_{x^*\in\dom f^*}|\expval{x,x^*}-f^*(x^*)|=\sup_{x\in X^*}\expval{x,x^*}-f^*(x^*)=f^{**}(x)$
\end{proof}
\begin{theorem}
	$X$ e.v.n. $f$ convexe: $X\rightarrow ]-∞,+∞]$ tel que $\exists x_0\in X\ |\ f(x)<+∞$. Alors
	\begin{enumerate}
		\item $f$ s.c.i. sur $X$ $\implies$ $f^{**}=f$
		\item supposons que $\dom f^*≠ø$ alors $f^{**}=\bar f$.
	\end{enumerate}
	où $f$ est la plus grande fonction s.c.i. inferiore à $G$  $(\bar f(x)=\sup\{g(x)\ |\ g s.c.i. ≤f\})=\inf_{x_n\to x}\liminf_{n\to ∞}f(x_n)=\lim_{r>0}(\inf_{B(u,r)}G))$
	
\end{theorem}
\begin{remark}
	$\epi\bar f=\overline{\epi f}$
\end{remark}
\begin{proof}[Preuve de ii)]
	\begin{rappel}
		$f^{**}≤f$ $\implies$ $f^{**}≤\bar f$
		$f^{**}(x)=\sup\{g(x)\ |\ g affine continue ≤f\}$
	\end{rappel}

	Soit $x_0\in X$ et $α_0<\bar f(x_0)$. Montrons que $α_0< f^{**}(x_0)$ (alors on aura $\bar f(x_0)≤f^{**}(x_0)$)
	
	Il suffit de montrer $g(x)=\expval{x,x_0^*}-β_0$ tel que $g≤f$ et $α_0≤g(x_0)$. $\bar f(x_0)>α_0$ $\implies$ $(x_0,α_0)\not\in \bar f$. convexe fermé de $X\times\R$.
	D'après Hahn-banach $\exists (x_0^*,β_0)\in X^*\times\R$ tel que $\expval{x_0,α_0}+α_0β_0<γ_0=\inf\{\expval{x,x_0^*}+αβ_0\}$
	
en particulier
$\expval{x_0,x_0^*}+α_0β_0<\expval{x,x^*}+αβ_0$ $\implies$ $β_0≤0$ (sinon si $f(x)<+∞$ on peut faire $α\to∞$).

$β_0<0$ impossible car ssi $\bar x$ est tel que $f(\bar x)<+∞$ alors $\expval{x_0,x_0^*}+α_0β_0<\expval{\bar x,x_0^*}+αβ_0\ \forall α>f(\bar x)$
qd $α\to +∞$ on obtient une contradiction. Donc $β_0≥0$.

1er cas $β_0>0$ On a: $\expval{x_0,x_0^*}+α_0β_0≤\expval{x,x_0^*}+β_0\bar f(x)$ ($α=\bar f(x)$) $\implies$ $\bar f(x)≥\underbrace{\expval{x_0-x,\frac{x_0^*}β}+α_0}_{g(x) affine continue}$
$g(x_0)=α_0$.
\end{proof}

$f:X\rightarrow ]-∞,+∞]$, $\epi f=\{(x,α)\in X\times\R\ f(x)≤α\}$

$\bar f(x)\overset{\text{def}}{=}\sup\{g(x)\ |\ g sci g≤f\}$

$(x_0,x_0^*)+α_0β_0<\inf\{\expval{x,x_0^+}+αβ_0\ |\ \bar f(x)≤α\}$

$f^{**}(x_0)\overset{?}{≥}\bar f(x_0), α_0<\bar f(x_0)$.

où $f^{**}(x_0)=\sup\{g(x_0)\ |\ g\text{---affine continue }g≤f\} β_0≥0$

$β_0>0$ OK
$f^{**}(x_0)≤α_0\ \forallα_0<\bar f(x_0)$ $\implies$ $f^{**}(x_0)≥\bar f(x_0)$.

$β_0=0$ on va montrer que $f^{**}(x_0)=+∞$
Par hypotese $\exists g_0$ affine continue telle que $g_0≤f$. 

Soit $ε>0$ et posons $γ_0=\inf\{\expval{x,x_0^*}+0\ |\ \bar f(x)≤α\}$ ($β_0=0$)

On soit que $\expval{x_0,x_0^*}<γ_0$ et $\expval{x,x^*_0}≥γ$ si $\bar f(x)<+∞$.

$\expval{x,x_0^*}+εf(x)≥γ_0+εg_0(x)$ $\forall x\in\dom f$ ($\bar f(x)≤f(x)<+∞$).

Donc $f(x)≥\underbrace{g_0(x)+\frac{γ_0-\expval{x,x_0^*}}ε}_{g_ε\text{ affine continue}}, \forall x\in X$.
$\implies$ $f^{**}(x_0)≥g_ε(x_0)=g_0(x_0)+\frac{γ_0-\expval{x_0,x_0^*}}ε, \forall ε>0$.

$γ_0>\expval{x_0,x_0^*}$ $\implies$ $\lim_{ε\to 0_+}g_ε(x_0)=+∞$ $\implies$ $f^{**}(x_0)=+∞$.

$f$ convexe et admet une minimisant affine continue $\implies$ $f^{**}=\bar f$.

\begin{remark}
	$f$ admet minimisante affine continue $\iff$ $\exists x_0^*\ |\ f^*(x_0)<+∞$
	($\dom f^*≠ø$).
\end{remark}

\begin{proof}
	i) $f$ est supposée convexe s.c.i. avec $\dom f≠ø$. Si $f$ admet une minimisante affine continue, d'apres ii), on aura $\bar f=f=f^{**}$. Montrons l'existence d'une telle minimisante.
	$\epi f$ est une convexe fermé de $X\times\R$ (car $f$ est convexe sci) qui est non vide car $\exists x_0\in X$ tel que $f(x_0)<+∞$. De plus $f(x_0)>-∞$ $\implies$ $\exists α_0<f(x_0)$ $\implies$ $(x_0,α_0)\in\epi f$. D'après HB strict: $\exists x_0^*\in X^*$, $\exists β_0\in \R | \expval{x_0,x_)^*}+α_0β_0<\inf\{\expval{x,x_0^*}+αβ_0\ |\ (x,α)\in \epi f\}$
	
	Alors $β_0≥0$.
	
	Si $β_0=0$, on doit aura $\expval{x_0,x_0^*}<\expval{x,x^*_0}$, $\forall x\in\dom f$. Pour $x=x_0$, on obtient une contradiction d'où $β_0>0$.
	
			Donc $β_0> 0$ et pour $α=f(x)$ $x\in\dom f$ $\expval{x_0,x_0^*}+α_0β_0≤\expval{x,x_0^*}+β_0f(x)$ $\implies$ $\forall x\in\dom f$, $f(x)≥\expval{x_0-x,x^*/β}+α_0$ $\implies$ $f≥g_0$ où $g_0$ affine continue $g_0(x)$.
\end{proof}
\begin{rappel}
	H.B. analogique $E$ espace vectoriel sur $\R$. $V$ sous espace vectoriel de $E$. $p:E\rightarrow [0,+∞[$ telle que $p(x+y)≤p(x)+p(y)$ $\forall (x,y)\in E^2$
	$p(λx)=λp(x)$, $\forall λ≥0 \forall x\in E$.
	
	$f:V\rightarrow \R$ forme linéaire telle que $f(x)≤p(x)\ \forall x\in V$
	
	Alors $\exists \tilde f$ forme lineaire sur $E$ telle que $\tilde f(x)=f(x)\ \forall x\in V \tilde f(x)≤p(x)\ \forall x\in E$. Si $p(-x)=p(x)$ (semi. norme) alors $|\tilde f(x)|≤p(x)$ $\forall x\in E$
	
	cas $E$=Hilbert $f$ continue sur $V$ fermé $|f(x)|≤C\norm{x}$ $\forall x\in V E=V\oplus V^\perp$. $\tilde f(x)=f(P_Vx)$ où $P_Vx=x^*$ $\iff$ $x^*\in V x-x^*\in V^\perp$
	
	$|\tilde f(x)|≤C\norm{P_Vx}≤C\norm{x}\ \forall x\in E$.
\end{rappel}

\begin{example}
	$C$ convexe de $E$. $0_E\in C j_C(x)=\inf_{t>0}\{\frac xt\in C\}$
	
	$C=\{\norm{x}≤1\}$
	
	$\norm{\frac xt}=\frac 1t\norm{x}≤1 \frac xt\in C$ $\iff$ $t≥\norm{x}$
	
	$j_c(x)=\norm{x}$
\end{example}
$\{f_c<1\}\subset C\subset \{j_C≤1\}$

$0<t<t'$ $\frac xt\in C$ $\implies$ $\frac x{t'}\in C$?

$\frac x{t'}=\frac t{t'}\frac xt+(1-\frac t{t'})0_E$.

$\mqty{\frac xt\in C\\0\in C}$ $\implies$ $\frac x{t'}\in C$

Donc $\{t>0\ |\ \frac xt\in C\} = [j_C(x),+∞[ ou ]j_C(x),+∞] \supset[1,+∞]$

\textbullet  $j_C(x)<1$
$\implies$$\{t>0\ |\ \frac xt\in C\}\supset [1,+∞[$ $\implies$ $x\in C$
\textbullet  $x\in C$ $\implies$ $1\in\{t>0\ |\ \frac xt\in C\}$ $\implies$ $j_C(x)≤ 1$

b) $j_C(x+y)≤j_C(x)+j_C(y)$
Soit $j_C(x)<α$ et $j_C(y)<β$ et montrons que $j_C(x+y)<α+β$ $j_C(x)<α+β$
$j_C(x)<α$ $\implies$ $\frac xα\in C et j_C(y)<β$ $\implies$ $\frac yβ\in C$

$\implies$ $\frac α{α+β}\frac xα +\frac β{α+β}\frac yβ\in C$
(convexité de $C$ et $0<\frac α{α+β}≤1$) 
$\implies$ $\frac{x+y}{α+β}\in C$
$\implies$ $j_C(x+y)≤α+β$
d'où la conclusion qd $α\downarrow j_C(x)$ et $β\downarrow j_C(y)$.

$j_C(λx)=λj_C(x)$ pour $λ≥0$ est évident

c) $0\in \mathring{C}$ $\iff$ $\exists M|_{j_C(x)}≤M\norm{x}$, $\forall x\in E$
(1)$\iff$(2)
(1) $\implies$ (2)
$0\in \mathring{C}$ $\implies$ $\exists r>0: \overline{B(0,r)}\subset C$ $\implies$ $\forall x≠0_E\ r\frac{x}{\norm{x}}\in C$ $\implies$ $\forall x≠0_E\ \frac xt\in C si t≥\frac{\norm{x}}r$
$\implies$ $j_C(\frac xt)<1 \forall t≥\norm{x}/r$ $\implies$ $j_C(\frac xt)≤1 \forall t≥\frac{\norm{x}}r$
$\implies$ $j_C(x)≤t \forall t≥\norm{x}/t$.
$\implies$ $j_C(x)≤\norm{x}r (M=\frac 1r)$
(2) $\implies$ (1) $j_C(x)≤M\norm{x}$ $\implies$ $j_C(x)<1$ si $\norm{x}<\frac 1M$ $\implies$ $\{\norm{x}<\frac 1M\}\subset C$ $\implies$ $0\in C$

d) $C$ ouvert $\iff$ $0\in \mathring{C}$ et $C=\{j_c<1\}$
(1)$\iff$(2)
(2)$\implies$(1) $0\in \mathring{C}$ $\implies$ $j_C(x)≤M\norm{x}$ $\implies$ $j_C$ convexe , $j_C$ continue sur toute boule $\implies$ $\{j_C<1\}$ est ouvert $\implies$ $C$ ouvert.

\textbullet  $j_C$$((1-θ)x+θy)≤j_C((1-θ)x)+j_C(θy)≤(1-θ)j_C(x)+θj_C(y)$ $\implies$ $j_C$ est convexe
sous additif et $1$ lienague 

(1) $\implies$ (2) $C$ ouvert $\implies$ $\mathring{C}=C$
(donc $0\in C$ $\implies$ $0\in \mathring{C}$)

On $0\in \mathring{C}$ $\implies$ $j_C$ continue $\implies$ $\{j_C<1\}$ est ouvert contenu dans $C$. Il reste à établir que $C\subset \{j_C<1\}$. $x\in C \{t>0\ |\ \frac xt\in C\}$

$x\in C$ $φ: t\rightarrow \frac xt$ continue de $]0,+∞[$ à valeurs dans $E$ $\implies$ $φ\dmo(C) =\{t>0\ |\ \frac xt\in C\}$

est un ouvert de $]0,+∞[$ comme image réciproque de l'ouvert $C$

Donc $\{t>0, \frac xt\in C\} x\in C$ $\implies$ $1 \in ]j_C(x),+∞[$
 $\implies$ $j_C(x)<1$
 Donc $C$ ouvert $\implies$ $\{j_C<1\} =C$
 
\begin{rappel}
	$f$ convexe s.c.i. $\implies$ $f^{**}=f$.
\end{rappel}

\begin{theorem}
	Soit $h:X\rightarrow ]-∞,+∞]$ convexe telle que h continue au point $x=0$.
	Alors
	\begin{enumerate}[(i)]
		\item $h^{*}$ atteint son minimum sur $X^*$ (et $arg\min h^*$ est *faiblement compact dans $X^{**}$)
		\item $\inf_{X^*}h^* (=\min_{X^*})=-h(0)$
	\end{enumerate}
\end{theorem}

\begin{remark}
	variante $h$ continue en $x_0$. Même résultat mais avec $x^*\rightarrow h^*(x^*)-\expval{x_0,x^*}$ et $\inf [h^*(•)-\expval{x_0,•}]=-h(x_0)$
	
	$\sup_x|\expval{x,x^*}-h(x_0+x)|=h^*(x^*)-\expval{x_0,x^*}$.
	$(h(x_0+•))^*=h^*-\expval{x_0,•}$
\end{remark}
\begin{proof}
	$h$ continue en $0$ $\implies$ $\exists R>0\ \exists M$> tels que $h(x)≤M$ $\forall\norm{x}<$R. Alors $h^*(x^*)=\sup_{x\in X}\expval{x,x^*}-h(x)≥\sup_{\norm{x}≤R}|\expval{x,x^*}-M|≥R\norm{x^*}_{X^*}-M$
	
	$\implies$ $\lim_{\norm{x^*}\to +∞}h^*(x^*)=+∞$.
	
	$\forall h^*$ est convexe sur $X^*$ $\forall R' \{\norm{x^*}≤R'\}$ est *faiblement compact $h^*$ est *faiblement s.c.i. 
	
	$\implies$ $\argmin_{X^*}h^*$ est un convexe non vide compact par la topologie faible.
	
	$K_n=\{h^*≤α_n\}$ compact $≠ø$ $α_n\downarrow \inf h^*$.
	
	Alors:
		$\argmin h^*=\{x^*\in X^* h^*(x^*)≤\inf h^*\} =\cap_n\{h^*≤α_n\}$
		
	Preuve de ii) $\inf h^*=-(h^*)^*(0)=-\bar h(0)$ $h$ continue en $0$ $\implies$ $\bar h(0)=h(0)$
\end{proof}
\section{Differentiabilite des fonctions convexes} % (fold)
\label{sec:differentiabilite_des_fonctions_convexes}
Soit $f:X\rightarrow ]-∞,+∞]$.

\begin{definition}
	Soit $x\in\dom f$ et $x^*\in X^*$. Alors $x^*$ est dans le sous-différentiel de $f$ au point $x$ si $\forall y\in X f(y)≥\underbrace{f(x)+\expval{y-x,x^*}}_{g(y)\text{ affine}}$.
\end{definition}

\begin{example}
	$f(x)=|x|$ sur $\R$. $\forall x^*\in[-1,1] x^*\in \pdv f()$. $|y|≥0+(y-0)x^*$, $y\in\R$. $x^*$ est an sous-gradient de $f$ au point $x$ $\pdv f(x)=\{x^*\in X^*\ |\ x^* sous gradient au point x\}$
	
	$f(x)=|x|$ au $\R$. $\pdv f(x)=\left\{ 
		\mqty{ \{\frac x{|x|}\} \text{ si }x≠0 \\ [-1,1]\text{ si }x=0}
	\right.$ 
\end{example}
\begin{lemme}
	\begin{enumerate}
		\item $x^*\in \pdv f(x)$ $\iff$$ \expval{x,x^*}=f(x)+f^*(x^*)$
		\item Si $f$ est convexe sci alors $x^*\in \pdv f(x)$ $\iff$ $x\in\pdv f^*(x^*)$
	\end{enumerate}
\end{lemme}
\begin{proof}
	$x^*\in \pdv f(x)$ $\iff$ $f(y)>f(x)+\expval{y-x,x^*}\ \forall y\in X$ $\iff$ $\expval{x,x^*}-f(x)≥\expval{y,x^*}-f(y)\ \forall y\in X$ $\iff$ $\expval{x,x^*}-f(x)≥\sup_{y\in X}\expval{y,x^*}-f(y)$ $\iff$ $\expval{x,x^*}≥f(x)+f^*(x^*)$ $\iff$ $"="$
	
	ii) $x\in \pdv f^*(x^*)$ $\iff$(i) $f^*(x^*)+\underbrace{f^{**}(x)}_{f(x)}=\expval{x,x^*}$ car $f$ convexe s.c.i.
\end{proof}
\begin{theorem}
	$f$ convexe $X\rightarrow ]-∞,+∞]$ tel que $\exists x_0\in\dom f | f$ continue en $x_0$. Alors $\forall x\in \mathcircle{\dom f} \pdv f(x)$ est une convexe compact non vide de $X^*$ (topologie *faible).
\end{theorem}
\begin{proof}
	on applique le Thm 2 avec $h(y)=f(x+y)$ où $x$ est un point de continuite de $f$ ($f$ continue au point $x$ $\iff$ $x\in \mathcircle{\dom f}$ et $\mathcircle{\dom f}≠ø$) Alors $h(0)=-\min h^*, h^*(x^*)=f^*(x^*)-\expval{x,x^*}$. $\implies$ $f(x)=-\min_{X^*}[f^*(x^*)-\expval{x,x^*}]$
	
	Si $x^*$ realese le minimum, on a:
	$f(x)=-f^*(x^*)+\expval{x,xt}$. Donc $x^*\i$$n \pdv f(x)$ $\iff$ $x^*\in\argmin[f^*-\expval{•,x^*}]$---compact non vide. 
\end{proof}

% section differentiabilite_des_fonctions_convexes (end)
\section{lien avec la différentialité} % (fold)
\label{sec:lien_avec_la_differentialite}
Soit $f:X\rightarrow \R$ continue dérivée de $f$ dans la direction $h\in X$ au point $x$:
$\lim_{ε\to 0_+}\frac{f(x+εh)-f(x)}ε\eqdef f'(x,h)$.

\begin{definition}
	$f$ est $G$-dérivable au point $x$ si $f'(x,h)$ existe et est lineaire continue par rapport à $h$. Notons $f'(x)$ l'eelement de $X^*$ associé à cette forme lineaire. 	 
\end{definition}
On a:
	$f'(x,h)=\expval{h,f'(x)} \forall h\in H. f'(x)$ est la $G$-dérivée de $f$ au point $x$.
	
	Si $f$ est de classe $C^1$ sur $X$ alors $f'(x)$ coïncide avec la différentielle de f noté $df(x)$ ou bien $\nabla f(x)$ si $X$ est un Hilbert
	
\begin{exercise}
	$f$ classe $C^1$ sur $\R^N \nabla f(x)=(\pdv{f}{x_1},\pdv{f}{x_2},...\pdv{f}{x_N})$
	$\lim_{t\to 0_+}\frac{f(x)+th)-f(x)}{t}=(\nabla f(x)|h)$
\end{exercise}

\begin{lemme}
	Soit $f:X\rightarrow ]-∞,+∞]$ convexe telle que $f$ est minorée par une function affine. Alors $\forall x\in\dom f$, $f'(x,h)$ existe $\forall h\in x$ et on a
	$f'(x,h)≥\expval{h,x^*}\ \forall h\in \pdv f(x)$.
\end{lemme}
\begin{proof}
	$\frac{f(x+εh)-f(x)}ε$ est une fonction monotone croissante de $ε$. Donc:
	$\lim_{ε\to 0_+}\frac{f(x+εh)-f(x)}ε=\inf_{ε>0}\frac{f(x+εh)-f(x)}ε$.
	
	Soit $x^*\in \pdv f(x)$ ($x\in \mathcircle{\dom f}$ $\implies$ $\pdv f(x)≠ø)$).Alors $f(x+εh)≥f(x)+\expval{εh,x^*}$ $\implies$ $\frac{f(x+εh)-f(x)}ε≥\expval{h,x^*}\forall ε$.
	
	Donc on a $f'(x,h)≥\expval{x^*,h}\ \forall x^*\in\pdv f(x)$. 
\end{proof}
\begin{corollary}
	Si $f$ est convexe, $G$ dérivable au point $x$, alors $\pdv f(x)=\{f'(x)\}$ où $f'(x)$ est la $G$ dérivée de $f$.
\end{corollary}
\begin{proof}
	$\expval{h,f'(x)}=f'(x,h)≥\expval{h,x^*}\ \forall x^*\in \pdv f(x)$
	
	Par linéarité, on a donc $f'(x)=x^*$.
\end{proof}
\begin{theorem}
	Si $f$ est convexe centre au point $x$. Alors $f$ est $G$ dérivable au point $x$ $\iff$ $\pdv f(x)$ est réduit à un point.
\end{theorem}
% section lien_avec_la_differentialite (end)
\section{Application de la dualité en optimisation} % (fold)
\label{sec:application_de_la_dualite_en_optimisation}
Soit $f:X\rightarrow ]-∞,+∞]$ convexe s.c.i. et $C$ un compacte fermé de $X$. On considère le Pb
$\inf_{x\in C}f(x)$.

On cherche des conditions nécessaires et suffisantes d'optimalité.
\begin{theorem}
	On supposons que $\exists x_0\in c$ tel que $f$ continue en $x_0$. Alors $\bar x$ solution $\iff$ $\exists x^*\in \pdv f(\bar x)$ tel que $\expval{y-\bar x,x^*}≥0\ \forall y\in C$.
\end{theorem}
\begin{proof}
	Soit $F=f+δ_C$.
	$F(x)=\left\{\smqty{f(x)\text{ si }x\in C\\+∞ \text{ sinon}}\right.$
	
	$f$ est convexe sci $\dom F≠0$. ($F(x_0)<+∞$). 
	
	$\bar x$ solution $\iff$ $F(\bar x)=\inf F=-F^*(0)$ $\iff$ $F(\bar x)+F^*(0)=\expval{\bar x,0}$ $\iff$ $0\in \pdv F(\bar x)$
	
	Supposons que: $\pdv F(\bar x)=\pdv (f+δ_C)(\bar x)=\pdv f(\bar x)+\pdv δ_C(\bar x)$
	
	Alors $\bar x$ solution $\iff$ $0\in \pdv f(\bar x)+\pdv δ_C(\bar x)$
	$\iff$ $\exists x^*\in \pdv f(\bar x)$ avec $-x^*\in\pdv δ_C(\bar x)$
	
	$-x^*\in\pdvδ_C(\bar x)$ $\iff$ $\underbrace{δ_C(\bar x)}_0+δ_C^*(-x^*)=-\expval{\bar x,x^*}$ $\iff$ $\sup_{y\in C}\expval{y,-x^*}=-\expval{\bar x,x^*}$
	$\iff$ $\forall y\in C -\expval{y,x^*}≤-\expval{\bar x,x^*}$ $\iff$ $\expval{y-\bar x,x^*}≥0 \forall y\in C$.
	
\end{proof}
\begin{theorem}
	$f,g$ convexes s.c.i. $X\rightarrow ]-∞,+∞]$ $\exists x_0\in\dom g | f$ continue au point $x_0$. Alors $\pdv(f+g)(x)=\pdv f(x)+\pdv g(x)$.
\end{theorem}
% section application_de_la_dualite_en_optimisation (end)
\section{Kuhn-Tucker} % (fold)
\label{sec:kuhn_tucker}
$C=\{x\in X\ |\ g_i(x)≤0, 1≤i≤N\}$ on $g_i$ affine continue sur $X$. ( $C$ déterminé par $N$ contraintes linéaires de type inégalité)

Soit $λ=(λ_1,λ_2,...,λ_N) \in\R_+^N$ et posons $F_X(x)=f(x)+∑_{i=1}^{i=N}λ_ig_i(x)$

$\sup_{λ≥0}F_λ(x)=f(x)$ si $x\in C$ et $+∞$ sinon.

On pénalise la contrante $C$ en remplacent $δ_C$ par $∑λ_ig_i(x)$.

\begin{theorem}
	Supposons qu'il existe x$_0\in\dom f$ tel que $g_i(x_0)<0$  $\forall i\in\{1,2,3,...,N\}$
	
	Condition de qualification SLATER) ($x_0\in\mathcircle{C}$).
	
	Alors $\bar x\in C$ solution $\iff$ $\exists \barλ\in(\R_+)^\N$ tel que:
	
	\begin{enumerate}
		\item $F_{\bar x}(\bar x)≤F_{\bar x}\ \forall x\in X$. ($\bar x$ réalise $\inf_XF_{\bar x}$)
		\item $\bar X_i g_i(\bar x)$ et $g_i(\bar x)≤0 \forall i\in\{1,...,N\}$
		($g_i(\bar x)<0$ $\implies$ $\bar λ_i=0$)
	\end{enumerate}
\end{theorem}
En pratique si $g_i(x)=\expval{x,x_i^*}+β_i$ et si $f$ est $G$ dérivable la condition équivalent à $F_{\bar x}'(\bar x)=0$
	
	$f'(\bar x)+∑_{i=1}^N\bar λ_ix_i^*=0$
	
$L(\bar x,x)≤L(\bar x,\bar λ)≤L(x,\bar λ)$
$\forall x\in X \forall λ\in R_N^+$
% section kuhn_tucker (end)
% chapter analyse_convexe (end)


\chapter{Exercices} % (fold)
\label{cha:exercices}
\section{exercice 1} % (fold)
\label{sec:exercice_1}

$\inf[∫_0^1f(u,u')\dd{x}| u(0)=0 u(1)=h]$
$f:\R\times\R \rightarrow [0,+∞]$
$X=\{u\in C^0([0,1]) |u'\in L^2(0,1)\}$
$F(u)=\mqty{∫_0^1 f(u,u')}\dd{x}$

$C=\{u\in X\ |\ u(0)=0\ u(1)=h\}$ convexe fermé dans X

$α|z|^2≤f(t,z)≤β(1+|z|^2)$ $\implies$ $\lim_{\norm{u}\to +∞}F(u)=+∞$ et $F$ est definie et continue sur $X$.
\begin{enumerate}
	\item Si $F$ est feulement s.c.i., alors $\exists\bar u\in C\ |\ F(\bar u)=\inf_X F$. (au $z\mapsto f(t,z)$ est convexe)
	But: caracteriser $\bar u$ à l'aide d'une équation. 
	\item  $\textbf{2)} f$ classe $C^1$ sur $\R\times\R_- u,v\in X$.

	$\lim_{ε\to 0}\frac{F(u+εv)-F(u)}ε$ ?
\end{enumerate}
	$∫\underbrace{[\frac{f(u+εv,u'+εv')-f(u,u')}{g_ε(x)}ε]}\dd{x}$
	
	$g_ε(x)\overset{pp}{\to}\pdv{f}{t}(u(x),u'(x))v(x)+\pdv{f}{z}(u(x),u'(x))v'(x)$
	
	$|g_ε(x)|≤h(x) où ∫_0^1h(x)\dd{x}<+∞$ $\implies$ $∫_0^1(\pdv{f}{t}(u,u')v(x)+\pdv{f}{z}(u,u')v'(x))\dd{x}$. Supposons que $F(\bar u)=\inf_C F$.
	
	Alors si $v\in X$ avec $v(0)=v(h)=0$, on a $u+εv\in C$ $((u+εv)(0)=0$ $(u+εv)(1)=u(1)=h)\forall ε>0$.
	
	Donc $F(\bar u+εv)≥F(\bar u)$, $\forall v\in X$ tel que $v(0)=v(1)=0$ $\implies$ $F'(\bar u, v)=0$, $\forall v\in X_0=\{v\in X\ |\ v(0)=v(1)=0\}$
	
	$∫_0^1[\pdv{f}{t}(\bar u,\bar u')v+\pdv{f}{t}(\bar u,\bar u')v']\dd{x}=0 \forall v\in X_0$.
	
	Supposons $v$ de classe $C^1$ et $v(0)=v(1)=0$
	$∫_0^1[\pdv{f}{t}(\bar u,\bar u')-(\pdv{f}{z}(\bar u,\bar u'))']v\dd{x}=0$
	
	$\implies$ $\bar u$ vérifie l'équation (Euler),
	
	$-\dv{x}(\pdv{f}{z}(\bar u,\bar u'))+\pdv{f}{t}(\bar u,\bar u')=0$
\textbf{1)}
\begin{example}
	f$(t,z)=\frac{|z|^2}2+g(t), g classe C^1 sur \R$
	$-\dv{x}[(\bar u)']+g'(\bar u)=0$
	$-\bar u''+g'(\bar u)=0$
	$g(t)=-t$ $\implies$ $\bar u''=-1$ $\implies$ $\bar u(x)=-\frac{x^2}2+λx
	-Δ\bar u+g'(\bar u)=0$
	
	$\inf\{\frac12∫_0^1|u'|^2-∫_0^1u\dd{x}\ |\ u(0)=0\ u(1)=h\}$
	$\implies$ $\bar u(x)=-\frac{x^2}2+(h+\frac 12)x$
\end{example}
\textbf{3)}
$\bar u$ denné $u_ε(y)=u(ψ_ε\dmo(y))$
où $ψ_ε$: $x\in [0,1]\rightarrow x+εφ(x)$ où $φ$ est $C^∞$ et $suppφ\subset]0,1[$

Si $ε$ est petit $ψ_ε$ est une bijection de $[0,1]$ sur $[0,1$$]$. (qui coïncide avec l'identité dans un voisinage de 0 et 1) $|ψ_ε(u)-1|≤Cε$ $\implies$ $ψ_ε'≥1-Cε>0$, si $ε<\frac 1C$ $\implies$ $ψ_ε\dmo existe$.

$u_ε'(y)=|u'|(ψ_ε\dmo(y))(ψ_ε\dmo)(ψ_ε\dmo)'(y)=\frac 1{ψ_ε'(ψ_ε'(y))}$

$F(u_ε)=∫_0^1 f(u_ε(y),u_ε'(y))\dd{y}=∫_0^1 f(\bar u(ψ_ε\dmo(y)),\frac{(\bar u)'(ψ_ε\dmo(y))}{ψ_ε\dmo(y)})\dd{y}$
Posons $x=ψ_ε\dmo(y)$ $(x+εφ(x)=y)$.

$∫_0^1 f(\bar u(x),\frac{\bar u'(x)}{1+εφ'(x)})(1+εφ'(x))\dd{x}$

$F(u_ε)=∫_0^1f(\bar u,\frac{\bar u'}{1+εφ'})\dd{x}+ ε∫f(\bar u,\frac{\bar u'}{1+εφ'})φ'\dd{x}$

$\frac{F(u_ε)-F(\bar u)}ε=∫_0^1\frac{f(\bar u,\frac{\bar u'}{1+εφ'})-f(\bar u,\bar u')}ε\dd{x}+∫_0^1f(\bar u, \frac{\bar u'}{1+εφ'})φ'\dd{x}
$
$\lim_{ε\to 0}\frac{f(\bar u,\frac{\bar u'}{1+εφ'})-f(\bar u,\bar u')}ε$ ?

$\sim \partial_z{ }f(\bar u,\bar u') \frac{\bar u'(\frac 1{1+εφ'}-1)}ε$

$\rightarrow 
-\partial_zf(\bar u,\bar u')\bar u'φ'$

$\lim_{ε\to 0}\frac{F(u_ε)-F(u)}ε=∫_0^1[-\partial_zf(\bar u,\bar u')\bar u'φ'+f(\bar u,\bar u')φ']\dd{x}$

$u_ε(0)=\bar u(ψ_ε\dmo(0))=\bar u(0)=0$
$u_ε(1)=\bar u(ψ_ε\dmo(1))=\bar u(1)=h$

$\implies$ $F(u_ε)≥F(\bar u)$

Donc $φ\in C^∞ φ(0)=φ(1)=0$ $\implies$ $∫_0^1[f(\bar u,\bar u')-\partial_zf(\bar u,\bar u')]φ'\dd{x}=0$

$\implies$ $∫_0^1\pdv{x} (\partial_zf(\bar u,\bar u')\bar u'-f(\bar u,\bar u'))φ=0$

$\implies$ $\exists λ\in\R\ |\ \partial_zf(\bar u,\bar u')\bar u' - f(u,\bar u')=N$

$\inf_{\substack{u(0)=h\\u(1)=0}}∫_0^1\sqrt{\frac{1+{u'}^2}{u}}\dd{x}$

$f(t,z)=\sqrt{\frac{1+|z|^2}t}$
$T=∫_0^1 \frac{\sqrt{1+{v'}^2 }}{\sqrt{2g(h-v)}} \dd{x}$ problème Bernoulli en 1690 résolu par Euler en 1740 et Lagrange en 1780.

% section exercice_1 (end)




% section biconjuguee_de_fenchel (end)
\section{exercice} % (fold)
\label{sec:exercice}

\begin{exercise}
	\begin{enumerate}[(i)]
		\item montrer que $\overline{\epi f}=\epi(\bar f)$
		\item $\bar f(x)=\lim_{r\to 0_+}\inf_{B(x,r)}f=\inf_{\substack{(x_n)}\\x_n\to x}\{\liminf_{n\to ∞}f(x_n)\}$
	\end{enumerate}
\end{exercise}

Etape 1: $\exists \tilde f:X\rightarrow ]-∞,+∞] | \overline{\epi f}=\epi\tilde f$ ($\tilde f$ est donc s.c.i.)

Etape 2: $\tilde f≤f$ et $g$ s.c.i. $≤f$ $\implies$ $g≤\tilde f$ $\implies$ $\tilde f=\bar f$.

$A\supset X\times\R$
\begin{enumerate}
	\item $(x,α)\in A$ $\implies$ $(x,β)\in A\ \forall β≥α$
	\item $\exists \tilde f:X\rightarrow ]-∞,+∞] | A=\epi\tilde f$.
\end{enumerate}

(1) $\iff$ (2)

$x\in X I(x)=\{α\in \R\ |\ (x,α)\in A\}$. D'après (1). $α\in I(x)$ $\implies$ $β\in I(x) \forall β≥α$. Posons $\tilde f(x)=\inf\{I(x)\}$. On a donc $I(x)=\left\{\mqty{[f(x),+∞]\\ \R\text{ si }\tilde f(x)=-∞}\right.$

$\epi \tilde f=\{(x,α)\ |\ \tilde f(x)≤α\}$

\paragraph{etape 1} % (fold)
\label{par:etape_1}
$\exists \tilde f$: $X\rightarrow ]-∞,+∞]\ |\ \overline{\epi f}=\epi \tilde f$ ($\tilde f$ est donc s.c.i.) $A=\overline{\epi f}$ est un fermé de $X\times\R$. Soit $(x,α)\in A$ et $β>α$. Alors $\exists (x_n,α_n)\in\epi f\ |\ x_n\to et α_n\toα$
$f(x_n)≤α_n\ \forall n$ $\implies$ $f(x_n)≤β$ pour $n$ grand $\implies$ $(χ_n,β)\in \epi f$ $\implies$ $(x,β)\in\overline{\epi f}$
donc $(x,β)\in A\ \forall β≥α$.

Alors $A=\epi\tilde f$
où $\tilde f(x)=\inf\{α\ |\ (x,α)\in\overline{\epi f}$

Conclusion: $\overline{\epi f}=\epi \tilde f$ et $\tilde f$ est s.c.i.

De plus $\epi f\subset\overline{\epi{f}}=\epi\tilde f$ $\implies$ $\tilde f≤f$.

Soit $g$ s.c.i. $≤f$. alors $\epi g\supset \epi f$.

$\left\{\mqty{\epi g\text{ fermé}\\\epi g\supset \epi f}\right.$ $\implies$ $\epi g\supset\overline{\epi f}=\epi(\tilde f)$ $\implies$ $f≤\tilde f$.

Donc on a bien
$\tilde f(x)=\sup\{g(x)\ |\ g s.c.i. ≤f\}\overset{\text{def}}{=} \bar f(x)$.
QED i)

Preuve de ii). Posons $h(x)=\lim_{r\to 0^+}\inf_{B(x,r)} f=\sup_{r>0}\inf_{B(x,r)}f$.

Alors $h(x)≤f(x)\ \forall x (car \inf_{B(x,r)}f≤f(x)\ \forall r>0)$

Posons $h_n(x)=\inf_{B(x,\frac1n)}f. Alors h(x)=\lim_{n\to ∞}h_n(x)=\sup_{n}h_n(x)$

$\inf\{f(y),\ \norm{y-x}<\frac 1n\}=h(x)≤α$
$h_n(x)≤α$, $\forall n$.

$\forall n\exists x_n\ |\ \norm{x_n-x}≤\frac 1n, f(x_n)≤α$ $\implies$ $(x_n,α)\in\epi f\ \forall n$ $\implies$ $(x,α)\in\overline{\epi f}$.

Donc $\epi h\subset\overset{\epi f}=\epi\tilde f$ $\implies$ $h≥\tilde f$

$(x,α)\in\epi\bar f$ $\implies$ $(x,α)\in\overbrace{\epi f}$ $\implies$ $\exists (x_n,α_n)\ |\ f(x_n)≤α_n$ ex $x_n\to x$, $α_n\to α$.

On a $\forall r>0$ $\inf_{B(x,r)}f≤α$. En effet $x_n\to x$ $\implies$ $\exists N | x_n\in B(x,r)\ \forall n≥N$ $\implies$ $f(x_n)≥\inf_{B(x,r)}f$, $\forall n≥N$ $\implies$ $α_n≥\norm{•} \forall n≥N$.

Donc $h(x)≤α$ i.e. $\overline{(x,α)}\in\epi h$.

Conclusion $\epi\bar f\subset \epi h\subset \epi\bar f$ $\implies$ $h=\bar f$.
% paragraph etape_1 (end)

% section exercice (end)
\section{exercice 2} % (fold)
\label{sec:exercise}
$j_C(x)=\inf\{t>0\ |\ \frac xt\in C\}$
$A$ ouvert non vide $(0\in A)$ $p(x)=j_A(x)$ sous additive 1 homogène.
$x_0≠0$
$V=\{tx_0,t\in \R\} f(tx_0)=t$

$f≤p$ sur $V$ ?
$f(tx_0)≤p(tx_0)$ car $t≤j_A(tx_0)$ $t≤0$ ok

$\forall t>0 j_A(tx_0)=tj_A(tx_0)>t$
car $j_A(x_0)>1$
D'après HB
$\exists \tilde f:X\rightarrow \R$ lineaire telle que $\tilde f=f$ sur $V$
$\tilde f(x)≤p(x)\ \forall x\in E$.

Puisque $0\in A$ ouvert, $\exists M>0 | j_A(x)≤M\norm{x}$ $\implies$ $\tilde f(x)≤M\norm{x}$ $\implies$ $\tilde f\in X^*$.

$\tilde f≤j_A$ sur $X$ $\implies$ 

$f(x)<1$ $\forall x\in A$

$A\in\{\tilde f≤1\}$ $\R$ espace affine fermé. $H=\{\tilde f=1\}$ hyperplan affine
$x_0≠0$
$V=\{tx_0, t\in\R\}$
$f(tx_0)=t$. Donc $f≤p$ sur $V$.

$A$ ouvert convexe $B$ convexe. $A\cap B≠ø$.

Posons $C=A-B=\{a-b\ |\ (a,b)\in A\times B\}$

$A\cup B=ø$ $\iff$ $0\not\in C$.
$A,B$ convexes $\implies$ $C$ est convexe 
$A$ ouvert $\implies$ $C$ ouvert, car $C=\cup_{b\in B}(A-\{b\})$

D'après 2) applique à $C$ et à $x_0=0$. $\exists f\in X^*$ tel que
$\sup_C f≤f(0)=0$

Or $\sup_Cf=\sup\{f(a-b)\ |\ (a,b)\in A\times B\}$
$C=\sup\{f(a)-f(b)\ |\ a\in A, b\in B\}=\sup_A f + \sup_B(-f) =\sup_{A}f-\inf_{B}f$.

$F$ fermé, $K$ compact $F\cap K=ø$.
$\exists ε>0 $tel que $F_ε\cap K_ε=ø$.

$\exists ε>0$ tel que $F_ε\cap K_ε=ø$
οù $F_ε=\{x\in X\ |\ d(x,F)<ε\}$ et $K_ε=\{x\in X\ |\ d(x,K)<ε\} $

$F_ε$ et $K_ε$ sont des ouverts car $d(•,F)$ et $d(•, K)$ sont continue.

$β=\inf_{\substack{y\in K\\z\in F}}\norm{y-z}>2ε$ $\implies$ $K_ε\cap F_ε=ø$.

$x\in F_ε\cap K_ε: d(x,F)<ε$ $\implies$ $\exists y\in F\ \norm{x-y}<ε et d(x,K)<ε$ $\implies$ $\exists y\in K\ \norm{x-y}<ε$

$\implies$ $\norm{y-z}<2ε$.

$A\cap B=ø$ $\iff$ $0\not\in C$. $A,B$ convexes $\implies$ $C$ est convexe. $A$ ouvert $\implies$ $C$ ouvert car $C=\cup_{b\in B}(A-\{b\})$

Supposons $β=0$. Alors $\exists (y_n,z_n)\in K\times F$. Tel que $\norm{y_n-z_n}\to 0$. 

$K$ compact $\implies$ $\exists \bar y\in K \exists n_k | y_{n_k}\to \bar y$ $\implies$ $z_{n_k}\to \bar y$

$F$ fermé $z_{n_k}\in F$ $\implies$ $\bar y\in f$ impossible $ε<\frac β2$ $\implies$ $K_ε\cap F_ε=ø$.

Donc $\exists f\in X^*$ tel que $\sup_{K_ε}≤\inf_{F_ε}f$.

$K_ε=\{x+εz\ |\ x\in K_ε, \norm{z}≤1\}$
$F_ε=\{x+εz'\ |\ x\in F_ε, \norm{z'}≤1\}$

$\sup_K f=\sup_{x\in K}\{f(x)-εf(z)\}≤\sup_K f+ε\norm{f}_{X^*}$

$\norm{z}<1$ 

$|\sup_{K_ε}f-\sup_Kf|≤ε\norm{f}_{X^*}$
$|\sup_{F_ε}f-\sup_Ff|≤ε\norm{f}_{X^*}$

$\exists α\in\R f(x)+εf(z)≤α≤f(y)+εf(z')$
$\forall x\in K\ \forall y\in F$
$\forall \norm{z}≤1, \forall \norm{z'}≤1$

$\implies$ $\forall x\in K f(x)≤α-ε\norm{f}$
$\forall y\in F f(y)≥α+ε\norm{f}$
$\implies$ $\sup_K ≤α-ε\norm{f}≤α+ε\norm{f}≤\inf_F f$
$\implies$ $\sup_K f<α<\inf_F f$

%\fi

% section exercise (end)
% chapter exercices (end)